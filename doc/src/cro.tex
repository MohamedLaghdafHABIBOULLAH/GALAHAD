\documentclass{galahad}

% set the package name

\newcommand{\package}{cro}
\newcommand{\packagename}{CRO}
\newcommand{\fullpackagename}{\libraryname\_\packagename}
\newcommand{\solver}{{\tt \fullpackagename\_crossover}}

\begin{document}

\galheader

%%%%%%%%%%%%%%%%%%%%%% SUMMARY %%%%%%%%%%%%%%%%%%%%%%%%

\galsummary
This package provides a {\bf crossover} from a solution
to the {\bf convex quadratic programming problem}
\disp{\mbox{minimize}\;\; q(\bmx) = \half \bmx^T \bmH \bmx + \bmg^T \bmx + f}
subject to the general linear constraints
\disp{c_{i}^{l}  \leq  \bma_{i}^{T}\bmx  \leq  c_{i}^{u}, \;\;\;
 i = 1, \ldots , m,}
and the simple bound constraints
\disp{x_{j}^{l}  \leq  x_{j}^{ } \leq  x_{j}^{u} , \;\;\; j = 1, \ldots , n,}
found by an interior-point method to one in which the
{\bf matrix of defining
active constraints/variables is of full rank.}
Here, the $n$ by $n$ symmetric, positive-semi-definite matrix $\bmH$, the
vectors $\bmg$, $\bma_{i}$, $\bmc^{l}$, $\bmc^{u}$, $\bmx^{l}$,
$\bmx^{u}$, the scalar $f$ are given. In addition a solution $\bmx$ along
with optimal Lagrange multipliers $\bmy$ for the general constraints
and dual variables $\bmz$ for the simple bounds must be provided
(see Section~\ref{galmethod}).
These will be adjusted as necessary.
Any of the constraint bounds $c_{i}^{l}$, $c_{i}^{u}$,
$x_{j}^{l}$ and $x_{j}^{u}$ may be infinite.
Full advantage is taken of any zero coefficients in the matrix $\bmH$
or the matrix $\bmA$ of vectors $\bma_{i}$.

%%%%%%%%%%%%%%%%%%%%%% attributes %%%%%%%%%%%%%%%%%%%%%%%%

\galattributes
\galversions{\tt  \fullpackagename\_single, \fullpackagename\_double}.
\galuses
{\tt GALAHAD\_CLOCK},
{\tt GALAHAD\_SY\-M\-BOLS},
{\tt GALAHAD\-\_SPACE},
{\tt GALAHAD\_\-SPECFILE},
{\tt GALAHAD\_TOOLS}.
{\tt GALAHAD\_QPT},
{\tt GALAHAD\_\-SCU},
{\tt GALAHAD\_\-SLS},
{\tt GALAHAD\_\-ULS},
\galdate January 2012.
\galorigin N. I. M. Gould, Rutherford Appleton Laboratory.
\gallanguage Fortran~95 + TR 15581 or Fortran~2003.
\galparallelism Some options may use OpenMP and its runtime library.

%%%%%%%%%%%%%%%%%%%%%% HOW TO USE %%%%%%%%%%%%%%%%%%%%%%%%

\galhowto

%\subsection{Calling sequences}

Access to the package requires a {\tt USE} statement such as

\medskip\noindent{\em Single precision version}

\hspace{8mm} {\tt USE \fullpackagename\_single}

\medskip\noindent{\em Double precision version}

\hspace{8mm} {\tt USE  \fullpackagename\_double}

\medskip

\noindent
If it is required to use both modules at the same time, the derived types
{\tt SMT\_type},
{\tt QPT\_problem\_type},
{\tt NLPT\_user\-data\-\_type},
{\tt \packagename\_time\_type},
{\tt \packagename\_control\_type},
{\tt \packagename\_inform\_type}
and
{\tt \packagename\_data\_type}
(Section~\ref{galtypes})
and the subroutines
{\tt \packagename\_initialize},
{\tt \packagename\_\-crossover},
{\tt \packagename\_terminate},
(Section~\ref{galarguments})
and
{\tt \packagename\_read\_specfile}
(Section~\ref{galfeatures})
must be renamed on one of the {\tt USE} statements.

%%%%%%%%%%%%%%%%%%%%%% parallel usage %%%%%%%%%%%%%%%%%%%%%%%%

%%%%%%%%%%%%%%%%%%%%%% parallel usage %%%%%%%%%%%%%%%%%%%%%%%%

\subsection{Parallel usage}
OpenMP may be used by the {\tt \fullpackagename} package to provide
parallelism for some solvers in shared memory environments.
See the documentation for the \galahad\ package {\tt SLS} for more details.
To run in parallel, OpenMP
must be enabled at compilation time by using the correct compiler flag
(usually some variant of {\tt -openmp}).
The number of threads may be controlled at runtime
by setting the environment variable {\tt OMP\_NUM\_THREADS}.

\noindent
MPI may also be used by the package to provide
parallelism for some solvers in a distributed memory environment.
To use this form of parallelism, MPI must be enabled at runtime 
by using the correct compiler flag
(usually some variant of {\tt -lmpi}).
Although the MPI process will be started automatically when required,
it should be stopped by the calling program once no further use of
this form of parallelism is needed. Typically, this will be via
statements of the form
\begin{verbatim}
      CALL MPI_INITIALIZED( flag, ierr )
      IF ( flag ) CALL MPI_FINALIZE( ierr )
\end{verbatim}

\noindent
The code may be compiled and run in serial mode.


%%%%%%%%%%%%%%%%%%%%%% derived types %%%%%%%%%%%%%%%%%%%%%%%%

\galtypes
Four derived data types are accessible from the package.

%%%%%%%%%%% control type %%%%%%%%%%%

\subsubsection{The derived data type for holding control
 parameters}\label{typecontrol}
The derived data type
{\tt \packagename\_control\_type}
is used to hold controlling data. Default values may be obtained by calling
{\tt \packagename\_initialize}
(see Section~\ref{subinit}),
while components may also be changed by calling
{\tt \packagename\_read\-\_specfile}
(see Section~\ref{readspec}).
The components of
{\tt \packagename\_control\_type}
are:

\begin{description}

\itt{error} is a scalar variable of type default \integer, that holds the
stream number for error messages. Printing of error messages in
{\tt \packagename\_crossover} and {\tt \packagename\_terminate}
is suppressed if
{\tt error} $\leq 0$.
The default is {\tt error = 6}.

\ittf{out} is a scalar variable of type default \integer, that holds the
stream number for informational messages. Printing of informational messages in
{\tt \packagename\_crossover} is suppressed if {\tt out} $< 0$.
The default is {\tt out = 6}.

\itt{print\_level} is a scalar variable of type default \integer, that is used
to control the amount of informational output which is required. No
informational output will occur if {\tt print\_level} $\leq 0$. If
{\tt print\_level} $= 1$, a single line of output will be produced for each
iteration of the process. If {\tt print\_level} $\geq 2$, this output will be
increased to provide significant detail of each iteration.
The default is {\tt print\_level = 0}.

\itt{max\_schur\_complement} is a scalar variable of type default \integer,
that specifies
the maximum number of columns permitted in the Schur complement when
updating the solution (see Section~\ref{galmethod})
before a re-factorization is triggered.
The default is {\tt max\_schur\_complement = 75}.

\itt{infinity} is a scalar variable of type default \realdp, that is used to
specify which constraint bounds are infinite.
Any bound larger than {\tt infinity} in modulus will be regarded as infinite.
The default is {\tt infinity =} $10^{19}$.

\itt{feasibility\_tolerance}
is a scalar variables of type default \realdp, that specifies the
maximum violation of the KKT conditions that is permitted.
The default iw {\tt feasibility\_toler\-ance =} $u$,
where $u$ is {\tt EPSILON(1.0)} ({\tt EPSILON(1.0D0)} in
{\tt \fullpackagename\_double}).

\itt{check\_io} is a scalar variable of type default \logical,
that must be set \true\ if the user wishes to check the input and output
data and  \false\ otherwise. The package may run faster if
{\tt space\_critical} is \false\ but at the possible expense of errors
The default is {\tt check\_io = .TRUE.}.

\itt{space\_critical} is a scalar variable of type default \logical,
that must be set \true\ if space is critical when allocating arrays
and  \false\ otherwise. The package may run faster if
{\tt space\_critical} is \false\ but at the possible expense of a larger
storage requirement. The default is {\tt space\_critical = .FALSE.}.

\itt{deallocate\_error\_fatal} is a scalar variable of type default \logical,
that must be set \true\ if the user wishes to terminate execution if
a deallocation  fails, and \false\ if an attempt to continue
will be made. The default is {\tt deallocate\_error\_fatal = .FALSE.}.

\itt{symmetric\_linear\_solver} is a scalar variable of type default \character\
and length 30, that specifies the external package to be used to
solve any symmetric linear system that might arise. Current possible
choices are {\tt 'sils'}, {\tt 'ma27'}, {\tt 'ma57'}, {\tt 'ma77'},
{\tt 'ma86'}, {\tt 'ma97'}, {\tt ssids}, {\tt 'pardiso'}
and {\tt 'wsmp'},
although only {\tt 'sils'} and, for OMP 4.0-compliant compilers,
{\tt 'ssids'} are installed by default.
See the documentation for the \galahad\ package {\tt SLS} for further details.
The default is {\tt symmetric\_linear\_solver = 'sils'},
but we recommend {\tt 'ma97'} if it available.

\itt{unsymmetric\_linear\_solver} is a scalar variable of type default
\character\
and length 30, that specifies the external package to be used to
solve any unsymmetric linear systems that might arise. Possible
choices are
{\tt 'gls'},
{\tt 'ma28'}
and
{\tt 'ma48'},
although only {\tt 'gls'} is installed by default.
See the documentation for the \galahad\ package {\tt ULS} for further details.
The default is {\tt unsymmetric\_linear\_solver = 'gls'},
but we recommend {\tt 'ma48'} if it available.

\itt{prefix} is a scalar variable of type default \character\
and length 30, that may be used to provide a user-selected
character string to preface every line of printed output.
Specifically, each line of output will be prefaced by the string
{\tt prefix(2:LEN(TRIM(prefix))-1)},
thus ignoring the first and last non-null components of the
supplied string. If the user does not want to preface lines by such
a string, they may use the default {\tt prefix = ""}.

\itt{SLS\_control} is a scalar variable argument of type
{\tt SLS\_control\_type} that is used to pass control
options to external packages used to solve any symmetric
linear systems that might arise.
See the documentation for the \galahad\ package {\tt SLS} for further details.
In particular, default values are as for {\tt SLS}.

\itt{ULS\_control} is a scalar variable argument of type
{\tt ULS\_control\_type} that is used to pass control
options to external packages used to solve any unsymmetric
linear systems that might arise.
See the documentation for the \galahad\ package {\tt ULS} for further details.
In particular, default values are as for {\tt ULS}.

\end{description}

%%%%%%%%%%% time type %%%%%%%%%%%

\subsubsection{The derived data type for holding timing
 information}\label{typetime}
The derived data type
{\tt \packagename\_time\_type}
is used to hold elapsed CPU and system clock times for the various parts of
the calculation. The components of
{\tt \packagename\_time\_type}
are:
\begin{description}
\itt{total} is a scalar variable of type default \realdp, that gives
 the total CPU time spent in the package.

\itt{analyse} is a scalar variable of type default \realdp, that gives
 the CPU time spent analysing the required matrices prior to factorization.

\itt{factorize} is a scalar variable of type default \realdp, that gives
 the CPU time spent factorizing the required matrices.

\itt{solve} is a scalar variable of type default \realdp, that gives
 the CPU time spent computing corrections to the current solution.

\itt{clock\_total} is a scalar variable of type default \realdp, that gives
 the total elapsed system clock time spent in the package.

\itt{clock\_analyse} is a scalar variable of type default \realdp, that gives
 the elapsed system clock time spent analysing the required matrices prior to
factorization.

\itt{clock\_factorize} is a scalar variable of type default \realdp, that gives
 the elapsed system clock time spent factorizing the required matrices.

\itt{clock\_solve} is a scalar variable of type default \realdp, that gives
 the elapsed system clock time spent computing corrections to
 the current solution.

\end{description}

%%%%%%%%%%% inform type %%%%%%%%%%%

\subsubsection{The derived data type for holding informational
 parameters}\label{typeinform}
The derived data type
{\tt \packagename\_inform\_type}
is used to hold parameters that give information about the progress and needs
of the algorithm. The components of
{\tt \packagename\_inform\_type}
are:

\begin{description}

\itt{status} is a scalar variable of type default \integer, that gives the
exit status of the algorithm.
%See Sections~\ref{galerrors} and \ref{galinfo}
See Section~\ref{galerrors}
for details.

\itt{alloc\_status} is a scalar variable of type default \integer, that gives
the status of the last attempted array allocation or deallocation.
This will be 0 if {\tt status = 0}.

\itt{bad\_alloc} is a scalar variable of type default \character\
and length 80, that  gives the name of the last internal array
for which there were allocation or deallocation errors.
This will be the null string if {\tt status = 0}.

\itt{dependent} is a scalar variable of type default \integer, that
gives the number of dependent active constraints.

\ittf{time} is a scalar variable of type {\tt \packagename\_time\_type}
whose components are used to hold elapsed CPU and system clock times for the
various parts of the calculation (see Section~\ref{typetime}).

\itt{SLS\_control} is a scalar variable argument of type
{\tt SLS\_control\_type} that is used to pass control
options to external packages used to solve any symmetric
linear systems that might arise.
See the documentation for the \galahad\ package {\tt SLS} for further details.
In particular, default values are as for {\tt SLS}.

\itt{ULS\_control} is a scalar variable argument of type
{\tt ULS\_control\_type} that is used to pass control
options to external packages used to solve any unsymmetric
linear systems that might arise.
See the documentation for the \galahad\ package {\tt ULS} for further details.
In particular, default values are as for {\tt ULS}.

\itt{scu\_status} is a scalar variable of type default \integer, that
gives the return status from the Schur-complement updating package
{\tt \libraryname\_SCU}.
See the specification sheet for {\tt \libraryname\_SCU} for details.

\itt{SCU\_inform} is a scalar variable of type
{\tt SCU\_info\_type}
whose components are used to provide information about
Schur-complement updating applied by the package
{\tt \libraryname\_SCU}.
See the specification sheet for {\tt \libraryname\_SCU} for details.

\end{description}

%%%%%%%%%%% data type %%%%%%%%%%%

\subsubsection{The derived data type for holding problem data}\label{typedata}
The derived data type
{\tt \packagename\_data\_type}
is used to hold all the data for a particular problem,
or sequences of problems with the same structure, between calls of
{\tt \packagename} procedures.
This data should be preserved, untouched, from the initial call to
{\tt \packagename\_initialize}
to the final call to
{\tt \packagename\_terminate}.

%%%%%%%%%%%%%%%%%%%%%% argument lists %%%%%%%%%%%%%%%%%%%%%%%%

\galarguments
There are three procedures for user calls
(see Section~\ref{galfeatures} for further features):

\begin{enumerate}
\item The subroutine
      {\tt \packagename\_initialize}
      is used to set default values, and initialize private data,
      before solving one or more problems with the
      same sparsity and bound structure.
\item The subroutine
      {\tt \packagename\_crossover}
      is called to solve the problem.
\item The subroutine
      {\tt \packagename\_terminate}
      is provided to allow the user to automatically deallocate array
       components of the private data, allocated by
       {\tt \packagename\_crossover},
       at the end of the solution process.
\end{enumerate}
We use square brackets {\tt [ ]} to indicate \optional\ arguments.

%%%%%% initialization subroutine %%%%%%

\subsubsection{The initialization subroutine}\label{subinit}
 Default values are provided as follows:
\vspace*{1mm}

\hspace{8mm}
{\tt CALL \packagename\_initialize( data, control, inform )}

\vspace*{-3mm}
\begin{description}

\itt{data} is a scalar \intentinout\ argument of type
{\tt \packagename\_data\_type}
(see Section~\ref{typedata}).
It is used to hold private data used by the crossover algorithm.

\itt{control} is a scalar \intentout\ argument of type
{\tt \packagename\_control\_type}
(see Section~\ref{typecontrol}).
On exit, {\tt control} contains default values for the components as
described in Section~\ref{typecontrol}.
These values should only be changed after calling
{\tt \packagename\_initialize}.

\itt{inform} is a scalar \intentout\ argument of type
{\tt \packagename\_inform\_type}
(see Section~\ref{typeinform}). A successful call to
{\tt \packagename\_initialize}
is indicated when the  component {\tt status} has the value 0.
For other return values of {\tt status}, see Section~\ref{galerrors}.

\end{description}

%%%%%%%%% main solution subroutine %%%%%%

\subsubsection{The crossover subroutine}\label{qps}
The crossover algorithm is called as follows:
\vspace*{1mm}

\hspace{8mm}
{\tt CALL \packagename\_crossover( n, m, m\_equal, H\_val, H\_col, H\_ptr, A\_val,     \&}
\vspace*{-1mm}

\hspace{43.5mm}
{\tt                               A\_col, A\_ptr, G, C\_l, C\_u, X\_l, X\_u, C, X, Y, \&}
\vspace*{-1mm}

\hspace{43.5mm}
{\tt                               Z, C\_stat, X\_stat, data, control, inform )}

%\vspace*{-3mm}
\begin{description}
\ittf{n} is a scalar \intentin\ argument of type default \integer,
 that must be set  to the number of optimization variables, $n$.
\restriction {\tt n} $> 0$.

\ittf{m} is a scalar \intentin\ argument of type default \integer,
 that must be set to the number of general linear constraints, $m$.
\restriction {\tt m} $\geq 0$.

\itt{m\_equal} is a scalar \intentin\ argument of type default \integer,
 that must be set to the number of general linear constraints that
 are equalities, i.e., whose lower and upper bounds coincide.
\restriction {\tt m} $\geq$ {\tt m\_equal} $> 0$.

\itt{H\_val} is an \intentin\  rank-one array argument
of dimension {\tt H\_ptr(n+1)-1} and type default \realdp, that must be set
to hold the values of the nonzero entries of the {\bf lower triangular} part
of the Hessian matrix $\bmH$. The entries must be ordered so that those
in row $i$ appear directly before those in row $i+1$ with no gaps;
the order within each row is unimportant.

\itt{H\_col} is an \intentin\ rank-one array argument
of dimension {\tt H\_ptr(n+1)-1} and type default \integer, that must be set
to hold the column indices of the {\bf lower triangular} part of
$\bmH$. These must be ordered so that they correspond to the values stored
in {\tt H\_val}.

\itt{H\_ptr} is an \intentin\ rank-one array argument
of dimension {\tt n+1} and type default \integer,
whose $j$-th component, $j = 1, \ldots , n$, holds the
starting position of row $j$ of the {\bf lower triangular} part of
$\bmH$ as stored in {\tt H\_val} and {\tt H\_col}. The $n+1$-st
component must be set to the total number of entries in the lower
triangular part of $\bmH$ plus one.

\itt{A\_val} is an \intentin\  rank-one array argument
of dimension {\tt A\_ptr(m+1)-1} and type default \realdp, that must be set
to hold the values of the nonzero entries
of the Jacobian matrix $\bmA$. The entries must be ordered so that those
in row $i$ appear directly before those in row $i+1$ with no gaps;
the order within each row is unimportant. 
\restriction the rows of $\bmA$ {\bf must be ordered} 
so that the first {\tt m\_equal} are equality constraints.

\itt{A\_col} is an \intentin\ rank-one array argument
of dimension {\tt A\_ptr(m+1)-1} and type default \integer, that must be set
to hold the column indices of $\bmA$. These must be ordered so that they
correspond to the values stored in {\tt A\_val}.

\itt{A\_ptr} is an \intentin\ rank-one array argument
of dimension {\tt m+1} and type default \integer,
whose $i$-th component, $i = 1, \ldots , m$, holds the  starting position
of row $i$ of $\bmA$ as stored in {\tt A\_val} and {\tt A\_col}.
The $m+1$-st component must be set to the total number of entries
in $\bmA$ plus one.

\ittf{G} is an \intentin\ rank-one array argument of dimension {\tt n}
and type default \realdp, that holds the gradient $\bmg$
of the linear term of the quadratic objective function.
The $j$-th component of {\tt G}, $j = 1,  \ldots ,  n$, must be set to
$\bmg_{j}$.

\ittf{C\_l} is an \intentin\ rank-one array argument of dimension {\tt m}
and type default \realdp, that holds the vector of lower bounds $\bmc^{l}$
on the general constraints. The $i$-th component of
{\tt C\_l}, $i = 1, \ldots , m$, must be set to $\bmc_{i}^{l}$.
Infinite bounds are allowed by setting the corresponding
components of {\tt C\_l} to any value smaller than {\tt -infinity},
where {\tt infinity} is a component of the control array {\tt control}
(see Section~\ref{typecontrol}).

\ittf{C\_u} is an \intentin\ rank-one array argument of dimension {\tt m}
and type default \realdp, that holds the vector of upper bounds $\bmc^{u}$
on the general constraints. The $i$-th component of
{\tt C\_u}, $i = 1,  \ldots ,  m$, must be set to $\bmc_{i}^{u}$.
Infinite bounds are allowed by setting the corresponding
components of {\tt C\_u} to any value larger than {\tt infinity},
where {\tt infinity} is a component of the control array {\tt control}
(see Section~\ref{typecontrol}).

\ittf{X\_l} is an \intentin\ rank-one array argument of dimension {\tt n}
and type default \realdp, that holds
the vector of lower bounds $\bmx^{l}$ on the the variables.
The $j$-th component of {\tt X\_l}, $j = 1, \ldots , n$,
must be set to $\bmx_{j}^{l}$.
Infinite bounds are allowed by setting the corresponding
components of {\tt X\_l} to any value smaller than {\tt -infinity},
where {\tt infinity} is a component of the control array {\tt control}
(see Section~\ref{typecontrol}).

\ittf{X\_u} is an \intentin\  rank-one array argument of dimension {\tt n}
and type default \realdp, that holds
the vector of upper bounds $\bmx^{u}$ on the variables.
The $j$-th component of {\tt X\_u}, $j = 1, \ldots , n$,
must be set to $\bmx_{j}^{u}$.
Infinite bounds are allowed by setting the corresponding
components of {\tt X\_u} to any value larger than that {\tt infinity},
where {\tt infinity} is a component of the control array {\tt control}
(see Section~\ref{typecontrol}).

\ittf{X} is an \intentinout\ rank-one array argument of dimension {\tt n}
and type default \realdp,
that holds the values $\bmx$ of the optimization variables.
The $j$-th component of {\tt X}, $j = 1,  \ldots , n$, must be set to $x_{j}$
on input, and may have been adjusted to provide another solution on output.

\ittf{C} is an \intentinout\ rank-one array argument of dimension {\tt m}
and type default \realdp, that holds
the values $\bmA \bmx$ of the constraints.
The $i$-th component of {\tt C}, $i = 1,  \ldots ,  m$, must be set to
$\bma_{i}^{T}\bmx \equiv (\bmA \bmx)_{i}$
on input, and may have been adjusted to provide another solution on output.

\ittf{Y} is an \intentinout\ rank-one array argument of dimension {\tt m}
and type default \realdp, that holds
the values $\bmy$ of estimates  of the Lagrange multipliers
corresponding to the general linear constraints (see Section~\ref{galmethod}).
The $i$-th component of {\tt Y}, $i = 1,  \ldots ,  m$, must be set to $y_{i}$
on input, and may have been adjusted to provide another solution on output.

\ittf{Z} is an \intentinout\ rank-one array argument of dimension {\tt n}
and type default \realdp, that holds
the values $\bmz$ of estimates  of the dual variables
corresponding to the simple bound constraints (see Section~\ref{galmethod}).
The $j$-th component of {\tt Z}, $j = 1,  \ldots ,  n$, must be set to $z_{j}$
on input, and may have been adjusted to provide another solution on output.

\itt{C\_stat} is an \intentin\ rank-one \intentinout\ array argument of
dimension {\tt m}
and type default \integer, that indicates which of the general linear
constraints are in the current active set. Possible input values for
{\tt C\_stat(i)}, {\tt i}$=1, \ldots ,$ {\tt m}, and their meanings are
\begin{description}
\itt{<0} the $i$-th general constraint
is in the active set on its lower bound, i.e., $\bma_i^T \bmx = \bmc_i^{l}$,
\itt{>0} the $i$-th general constraint
is in the active set on its upper bound, i.e., $\bma_i^T \bmx = \bmc_i^{u}$,
and
\itt{0}  the $i$-th general constraint is not in the active set,
i.e., $\bmc_i^{l} < \bma_i^T \bmx < \bmc_i^{u}$.
\end{description}
On output, the value of {\tt C\_stat(i)} may have changed to mean
\begin{description}
\itt{-1} the $i$-th general constraint
is both independent and active on its lower bound,
\itt{-2} the $i$-th general constraint
is on its lower bound but linearly dependent on others,
\itt{1} the $i$-th general constraint
is both independent and active on its upper bound,
\itt{2} the $i$-th general constraint
is on its upper bound but linearly dependent on others, and
\itt{0} the $i$-th general constraint is not in the active set.
\end{description}


\itt{X\_stat} is an \intentin\ rank-one \intentinout\ array argument of
dimension {\tt n}
and type default \integer, that indicates which of the simple bound
constraints are in the current active set. Possible input values for
{\tt X\_stat(j)}, {\tt j}$=1, \ldots ,$ {\tt n}, and their meanings are
\begin{description}
\itt{<0} the $j$-th simple bound constraint
is in the active set on its lower bound, i.e., $\bmx_j^{} = \bmx_j^{l}$,
\itt{>0} the $j$-th simple bound constraint
is in the active set on its upper bound, i.e., $\bmx_j^{} = \bmx_j^{u}$,
and
\itt{0}  the $j$-th simple bound constraint is not in the active set,
i.e., $\bmx_j^{l} < \bmx_j^{} < \bmx_j^{u}$.
\end{description}
On output, the value of {\tt X\_stat(j)} may have changed to mean
\begin{description}
\itt{-1} the $j$-th simple bound constraint
is both independent and active on its lower bound,
\itt{-2} the $j$-th simple bound constraint
is on its lower bound but linearly dependent on others,
\itt{1} the $j$-th simple bound constraint
is both independent and active on its upper bound,
\itt{2} the $j$-th simple bound constraint
is on its upper bound but linearly dependent on others, and
\itt{0} the $j$-th simple bound constraint is not in the active set.
\end{description}

\itt{data} is a scalar \intentinout\ argument of type
{\tt \packagename\_data\_type}
(see Section~\ref{typedata}).
It is used to hold private data used by the crossover algorithm
and must not have been altered {\bf by the user} since the last call to
{\tt \packagename\_initialize}.

\itt{control} is a scalar \intentin\ argument of type
{\tt \packagename\_control\_type}
(see Section~\ref{typecontrol}). Default values may be assigned by calling
{\tt \packagename\_initialize} prior to the first call to
{\tt \packagename\_crossover}.

\itt{inform} is a scalar \intentinout\ argument of type
{\tt \packagename\_inform\_type}
(see Section~\ref{typeinform}).
A successful call to
{\tt \packagename\_crossover}
is indicated when the  component {\tt status} has the value 0.
For other return values of {\tt status}, see Section~\ref{galerrors}.

\end{description}

%%%%%%% termination subroutine %%%%%%

\subsubsection{The  termination subroutine}
All previously allocated arrays are deallocated as follows:
\vspace*{1mm}

\hspace{8mm}
{\tt CALL \packagename\_terminate( data, control, inform )}

\vspace*{-3mm}
\begin{description}

\itt{data} is a scalar \intentinout\ argument of type
{\tt \packagename\_data\_type}
exactly as for
{\tt \packagename\_crossover},
that must not have been altered {\bf by the user} since the last call to
{\tt \packagename\_initialize}.
On exit, array components will have been deallocated.

\itt{control} is a scalar \intentin\ argument of type
{\tt \packagename\_control\_type}
exactly as for
{\tt \packagename\_crossover}.

\itt{inform} is a scalar \intentout\ argument of type
{\tt \packagename\_inform\_type}
exactly as for
{\tt \packagename\_crossover}.
Only the component {\tt status} will be set on exit, and a
successful call to
{\tt \packagename\_terminate}
is indicated when this  component {\tt status} has the value 0.
For other return values of {\tt status}, see Section~\ref{galerrors}.

\end{description}


%%%%%%%%%%%%%%%%%%%%%% Warning and error messages %%%%%%%%%%%%%%%%%%%%%%%%

\galerrors
A negative value of {\tt inform\%status} on exit from
{\tt \packagename\_crossover}
or
{\tt \packagename\_terminate}
indicates that an error has occurred. No further calls should be made
until the error has been corrected. Possible values are:

\begin{description}

\itt{\galerrallocate.} An allocation error occurred. A message indicating
the offending
array is written on unit {\tt control\%error}, and the returned allocation
status and a string containing the name of the offending array
are held in {\tt inform\%alloc\_\-status}
and {\tt inform\%bad\_alloc} respectively.

\itt{\galerrdeallocate.} A deallocation error occurred.
A message indicating the offending
array is written on unit {\tt control\%error} and the returned allocation
status and a string containing the name of the offending array
are held in {\tt inform\%alloc\_\-status}
and {\tt inform\%bad\_alloc} respectively.

\itt{\galerrrestrictions.} One of the restrictions
 {\tt prob\%n} $> 0$
    or the requirement that
    {\tt prob\%H\_type} contain its relevant string
    {\tt 'DENSE'}, {\tt 'COORDINATE'}, {\tt 'SPARSE\_BY\_ROWS'}
    or {\tt 'DIAGONAL'} when $\bmH$ is available,
    has been violated.

%\itt{-3.} At least one of the arrays
% {\tt prob\%A\_val}, {\tt prob\%A\_row}, {\tt prob\%A\_col},
% {\tt prob\%H\_val}, {\tt prob\%H\_row} or {\tt prob\%H\_col},
% is not large enough to hold the original, or reordered, matrix $\bmH$.

\itt{\galerrbadbounds.} The bound constraints are inconsistent.

\itt{\galerrunbounded.} The objective function appears to be unbounded
  from below on the feasible set.

\itt{\galerranalysis.} An error was reported by {\tt SLS\_analyse}. The return
status from {\tt SLS\_\-analyse} is given in
{\tt inform\%SLS\_inform\%\-status}.
See the documentation for the \galahad\ package {\tt SLS} for further details.

\itt{\galerrfactorization.} An error was reported by {\tt SLS\_factorize} or
{\tt SCU\_append}.
The return status from {\tt SLS\_factorize} is given in
{\tt inform\%SLS\_inform\%status}
and that from {\tt SCU\_append} in {\tt inform\%scu\_status}.
See the documentation for the \galahad\ packages {\tt SLS} and {\tt SCU}
for further details.

\itt{\galerrsolve.} An error was reported by {\tt SLS\_solve} or
{\tt SCU\_solve}. The return
status from {\tt SLS\_solve} is given in {\tt inform\%SLS\_\-inform\%status}
and that from {\tt SCU\_solve} in {\tt inform\%scu\_status}.
See the documentation for the \galahad\ packages {\tt SLS} and {\tt SCU}
for further details.

\itt{\galerrulsfactorization.} An error was reported by {\tt ULS\_factorize}.
The return status from {\tt ULS\_factorize} is given in
{\tt inform\%uls\_\-factorize\_\-status}.
See the documentation for the \galahad\ package {\tt ULS} for further details.

\itt{\galerrulssolve.} An error was reported by {\tt ULS\_solve}. The return
status from {\tt ULS\_solve} is given in {\tt inform\%uls\_solve\_status}.
See the documentation for the \galahad\ package {\tt ULS} for further details.

\end{description}

%%%%%%%%%%%%%%%%%%%%%% Further features %%%%%%%%%%%%%%%%%%%%%%%%

\galfeatures
\noindent In this section, we describe an alternative means of setting
control parameters, that is components of the variable {\tt control} of type
{\tt \packagename\_control\_type}
(see Section~\ref{typecontrol}),
by reading an appropriate data specification file using the
subroutine {\tt \packagename\_read\_specfile}. This facility
is useful as it allows a user to change  {\tt \packagename} control parameters
without editing and recompiling programs that call {\tt \packagename}.

A specification file, or specfile, is a data file containing a number of
"specification commands". Each command occurs on a separate line,
and comprises a "keyword",
which is a string (in a close-to-natural language) used to identify a
control parameter, and
an (optional) "value", which defines the value to be assigned to the given
control parameter. All keywords and values are case insensitive,
keywords may be preceded by one or more blanks but
values must not contain blanks, and
each value must be separated from its keyword by at least one blank.
Values must not contain more than 30 characters, and
each line of the specfile is limited to 80 characters,
including the blanks separating keyword and value.



The portion of the specification file used by
{\tt \packagename\_read\_specfile}
must start
with a "{\tt BEGIN \packagename}" command and end with an
"{\tt END}" command.  The syntax of the specfile is thus defined as follows:
\begin{verbatim}
  ( .. lines ignored by CRO_read_specfile .. )
    BEGIN CRO
       keyword    value
       .......    .....
       keyword    value
    END
  ( .. lines ignored by CRO_read_specfile .. )
\end{verbatim}
where keyword and value are two strings separated by (at least) one blank.
The ``{\tt BEGIN \packagename}'' and ``{\tt END}'' delimiter command lines
may contain additional (trailing) strings so long as such strings are
separated by one or more blanks, so that lines such as
\begin{verbatim}
    BEGIN CRO SPECIFICATION
\end{verbatim}
and
\begin{verbatim}
    END CRO SPECIFICATION
\end{verbatim}
are acceptable. Furthermore,
between the
``{\tt BEGIN \packagename}'' and ``{\tt END}'' delimiters,
specification commands may occur in any order.  Blank lines and
lines whose first non-blank character is {\tt !} or {\tt *} are ignored.
The content
of a line after a {\tt !} or {\tt *} character is also
ignored (as is the {\tt !} or {\tt *}
character itself). This provides an easy manner to "comment out" some
specification commands, or to comment specific values
of certain control parameters.

The value of a control parameters may be of three different types, namely
integer, logical or real.
Integer and real values may be expressed in any relevant Fortran integer and
floating-point formats (respectively). Permitted values for logical
parameters are "{\tt ON}", "{\tt TRUE}", "{\tt .TRUE.}", "{\tt T}",
"{\tt YES}", "{\tt Y}", or "{\tt OFF}", "{\tt NO}",
"{\tt N}", "{\tt FALSE}", "{\tt .FALSE.}" and "{\tt F}".
Empty values are also allowed for
logical control parameters, and are interpreted as "{\tt TRUE}".

The specification file must be open for
input when {\tt \packagename\_read\_specfile}
is called, and the associated device number
passed to the routine in device (see below).
Note that the corresponding
file is {\tt REWIND}ed, which makes it possible to combine the specifications
for more than one program/routine.  For the same reason, the file is not
closed by {\tt \packagename\_read\_specfile}.

\subsubsection{To read control parameters from a specification file}
\label{readspec}

Control parameters may be read from a file as follows:
\hskip0.5in

\def\baselinestretch{0.8}
{\tt
\begin{verbatim}
     CALL CRO_read_specfile( control, device )
\end{verbatim}
}
\def\baselinestretch{1.0}

\begin{description}
\itt{control} is a scalar \intentinout argument of type
{\tt \packagename\_control\_type}
(see Section~\ref{typecontrol}).
Default values should have already been set, perhaps by calling
{\tt \packagename\_initialize}.
On exit, individual components of {\tt control} may have been changed
according to the commands found in the specfile. Specfile commands and
the component (see Section~\ref{typecontrol}) of {\tt control}
that each affects are given in Table~\ref{specfile}.

\bctable{|l|l|l|}
\hline
  command & component of {\tt control} & value type \\
\hline
  {\tt error-printout-device} & {\tt \%error} & integer \\
  {\tt printout-device} & {\tt \%out} & integer \\
  {\tt print-level} & {\tt \%print\_level} & integer \\
  {\tt maximum-dimension-of-schur-complement} & {\tt \%max\_schur\_complement} & integer \\
  {\tt infinity-value} & {\tt \%infinity} & real \\
  {\tt feasibility-tolerance} & {\tt \%feasibility\_tol} & real \\
  {\tt check-input-output}   & {\tt \%check\_io} & logical \\
  {\tt space-critical}   & {\tt \%space\_critical} & logical \\
  {\tt deallocate-error-fatal}   & {\tt \%deallocate\_error\_fatal} & logical \\
  {\tt output-line-prefix} & {\tt \%prefix} & character \\
  {\tt symmetric-linear-equation-solver} & {\tt \%symmetric\_linear\_solver} & character \\
  {\tt unsymmetric-linear-equation-solver} & {\tt \%unsymmetric\_linear\_solver} & character \\
\hline

\ectable{\label{specfile}Specfile commands and associated
components of {\tt control}.}

\itt{device} is a scalar \intentin argument of type default \integer,
that must be set to the unit number on which the specfile
has been opened. If {\tt device} is not open, {\tt control} will
not be altered and execution will continue, but an error message
will be printed on unit {\tt control\%error}.

\end{description}

%%%%%%%%%%%%%%%%%%%%%% Information printed %%%%%%%%%%%%%%%%%%%%%%%%

\galinfo
If {\tt control\%print\_level} is positive, information about the progress
of the algorithm will be printed on unit {\tt control\-\%out}.
If {\tt control\%print\_level} $= 1$, a single line indicating how many
dependent constraints will be removed.
If {\tt control\%print\_level} $\geq 2$, this
output will be increased to provide details of the dependent constraints,
while if {\tt control\%print\_level} $\geq 5$, full debugging details
(probably only of interest to the code developer) are provided.

%%%%%%%%%%%%%%%%%%%%%% GENERAL INFORMATION %%%%%%%%%%%%%%%%%%%%%%%%

\galgeneral

\galcommon None.
\galworkspace Provided automatically by the module.
\galroutines None.
\galmodules {\tt \packagename\_crossover} calls the \galahad\ packages
{\tt GALAHAD\_CLOCK},
{\tt GALAHAD\_SY\-M\-BOLS},
{\tt GALAHAD\-\_SPACE},
{\tt GALAHAD\_\-SPECFILE},
{\tt GALAHAD\_TOOLS}.
{\tt GALAHAD\_QPT},
{\tt GALAHAD\_\-SCU},
{\tt GALAHAD\_\-SLS}
and
{\tt GALAHAD\_\-ULS}.
\galio Output is under control of the arguments
 {\tt control\%error}, {\tt control\%out} and {\tt control\%print\_level}.
\galrestrictions {\tt n} $> 0$, {\tt m} $\geq  m\_equal$,
{\tt m\_equal} $\geq  0$,
{\tt prob\%A\_type} and {\tt prob\%H\_type} $\in \{${\tt 'DENSE'},
 {\tt 'COORDINATE'}, {\tt 'SPARSE\_BY\_\-ROWS'}, {\tt 'DIAGONAL'} $\}$.
(if $\bmH$ and $\bmA$ are explicit).
\galportability ISO Fortran~95 + TR 15581 or Fortran~2003.
The package is thread-safe.

%%%%%%%%%%%%%%%%%%%%%% METHOD %%%%%%%%%%%%%%%%%%%%%%%%

\galmethod
Any required solution $\bmx$ necessarily satisfies
the primal optimality conditions
\eqn{4.1a}{\bmA \bmx = \bmc}
and
\eqn{4.1b}{
 \bmc^{l} \leq \bmc \leq \bmc^{u}, \;\;
\bmx^{l} \leq\bmx \leq\bmx^{u},}
the dual optimality conditions
\eqn{4.2a}{
 \bmH \bmx + \bmg =
 \bmA^{T} \bmy + \bmz,\;\;
 \bmy = \bmy^{l} + \bmy^{u} \tim{and}
 \bmz = \bmz^{l} + \bmz^{u} ,}
and
\eqn{4.2b}{
 \bmy^{l} \geq 0 , \;\;
 \bmy^{u} \leq 0 , \;\;
 \bmz^{l} \geq 0 \;\; \mbox{and} \;\;
 \bmz^{u} \leq 0 ,}
and the complementary slackness conditions
\eqn{4.3}{
( \bmA \bmx - \bmc^{l} )^{T} \bmy^{l} = 0  ,\;\;
( \bmA \bmx - \bmc^{u} )^{T} \bmy^{u} = 0  ,\;\;
(\bmx -\bmx^{l} )^{T} \bmz^{l} = 0   \tim{and}
(\bmx -\bmx^{u} )^{T} \bmz^{u} = 0 ,}
where the vectors $\bmy$ and $\bmz$ are
known as the Lagrange multipliers for
the general linear constraints, and the dual variables for the bounds,
respectively, and where the vector inequalities hold componentwise.

Denote the active constraints by $\bmA_A \bmx = \bmc_A$ and the
active bounds by $\bmI_A \bmx = \bmx_A$. Then any optimal solution satisfies
the linear system
\disp{
 \mat{ccc}{\bmH & - \bmA_A^T & - \bmI^T_A \\ \bmA_A & 0 & 0 \\ \bmI_A & 0 & 0 }
 \vect{\bmx \\ \bmy_A \\ \bmz_A} = \vect{ - \bmg \\ \bmc_A \\ \bmx_A }, }
where $\bmy_A$ and $\bmz_A$ are the corresponding active Lagrange multipliers
and dual variables respectively. Consequently the difference between
any two solutions $(\Delta \bmx, \Delta \bmy, \Delta \bmz)$ must satisfy
\eqn{difsol}{
 \mat{ccc}{\bmH & - \bmA_A^T & - \bmI^T_A \\ \bmA_A & 0 & 0 \\ \bmI_A & 0 & 0 }
 \vect{\Delta \bmx \\ \Delta \bmy_A \\ \Delta \bmz_A} = 0.}
Thus there can only be multiple solution if the coefficient matrix $\bmK$
of \req{difsol} is singular. The algorithm used in \fullpackagename\
exploits this. The matrix $\bmK$ is checked for singularity
using the \galahad\ package {\tt GALAHAD\_ULS}. If $\bmK$ is
non singular, the solution is unique and the solution input by the user
provides a linearly independent active set. Otherwise $\bmK$ is singular,
and partitions $\bmA_A^T = ( \bmA_{AB}^T \;\; \bmA_{AN}^T)$ and
$\bmI_A^T = ( \bmI_{AB}^T \;\; \bmI_{AN}^T)$ are found so that
\disp{
 \mat{ccc}{\bmH & - \bmA_{AB}^T & - \bmI^T_{AB} \\ \bmA_{AB} & 0 & 0 \\
 \bmI_{AB} & 0 & 0 }}
is non-singular and the ``non-basic'' constraints $\bmA_{AN}^T$
and $\bmI_{AN}^T$ are linearly dependent on the ``basic'' ones
$( \bmA_{AB}^T \;\; \bmI_{AB}^T)$. In this case \req{difsol} is equivalent to
\eqn{difsol2}{
 \mat{ccc}{\bmH & - \bmA_{AB}^T & - \bmI^T_{AB} \\
 \bmA_{AB} & 0 & 0 \\ \bmI_{AB} & 0 & 0 }
 \vect{\Delta \bmx \\ \Delta \bmy_{AB} \\ \Delta \bmz_{AB}} =
 \vect{\bmA_{AN}^T \\ 0 \\ 0 } \Delta \bmy_{AN} +
 \vect{\bmI^T_{AN} \\ 0 \\ 0 }\Delta \bmz_{AN} .}
Thus, starting from the user's $(\bmx, \bmy, \bmz)$
and with a factorization of the coefficient matrix of \req{difsol2}
found by the \galahad\ package {\tt GALAHAD\_SLS},
the alternative solution
$(\bmx + \alpha \bmx, \bmy + \alpha \bmy, \bmz + \alpha \bmz)$,
featuring
$(\Delta \bmx, \Delta \bmy_{AB}, \Delta \bmz_{AB})$
from \req{difsol2}
in which successively one of the components of $\Delta \bmy_{AN}$
and $\Delta \bmz_{AN}$ in turn is non zero, is taken.
The scalar $\alpha$ at each stage
is chosen to be the largest possible that guarantees \req{4.2b};
this may happen when a non-basic multiplier/dual variable reaches zero,
in which case the corresponding constraint is disregarded, or when this
happens for a basic multiplier/dual variable, in which case this constraint is
exchanged with the non-basic one under consideration and disregarded.
The latter corresponds to changing the basic-non-basic partition
in \req{difsol2}, and subsequent solutions may be found by updating
the factorization of the coefficient matrix in \req{difsol2}
following the basic-non-basic swap using the
\galahad\ package {\tt GALAHAD\_SCU}.

%%%%%%%%%%%%%%%%%%%%%% EXAMPLE %%%%%%%%%%%%%%%%%%%%%%%%

\galexample
Suppose we have solved the quadratic program
\disp{\arr{rl}{
\minin{\bmx} & \half \bigsum_{i=1}^{11} x_i^2 +
  \half \bigsum_{i=1}^{10} x_i x_{i+1}
+ \half x_1 - \half x_2 - \bigsum_{i=1}^{10} x_i - \half x_{11} \\
\mbox{subject to} & \bigsum_{i=1}^{11} x_i = 10, \;\;
\bigsum_{i=3}^{11} x_i \geq 9, \;\;
\bigsum_{i=2}^{11} x_i \leq 10 \\
\tim{and} & x_1 \geq 0, \;\; x_i \geq 1 \tim{for} i = 2, \ldots, 11
}}
(using, for example, \galahad's {\tt CQP} package), and have found
the primal-dual solution $\bmx = (0,1,1, \ldots, 1)$,
$\bmy = ( -1, \threehalves, -2)$ and
$\bmz = ( 2, 4, \fivehalves, \fivehalves, \ldots \fivehalves)$
for which all variables and constraints are active; clearly such
a solution has dependent active constraints.
Then we may find a crossover solution in which the defining
active set is linearly independent using the following code:

{\tt \small
\VerbatimInput{\packageexample}
}
\noindent
This produces the following output:
{\tt \small
\VerbatimInput{\packageresults}
}
\noindent
Notice that active variable 1 and constraints 2 and 3 are found to be active but
linearly dependent.
\end{document}

