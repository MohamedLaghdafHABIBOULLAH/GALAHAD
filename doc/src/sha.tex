\documentclass{galahad}

% set the package name

\newcommand{\package}{sha}
\newcommand{\packagename}{SHA}
\newcommand{\fullpackagename}{\libraryname\_\packagename}
\newcommand{\solver}{{\tt \fullpackagename\_analyse}}

\begin{document}

\galheader

%%%%%%%%%%%%%%%%%%%%%% SUMMARY %%%%%%%%%%%%%%%%%%%%%%%%

\galsummary
This package 
{\bf computes a component-wise secant approximation to the Hessian matrix}
$\bmH(\bmx)$, for which  
$(\bmH(\bmx))_{i,j} = \partial f^2 (\bmx) / \partial x_i \partial x_j$,
$1 \leq i, j \leq n$,  
using values of the gradient $\bmg(\bmx) = \nabla_x f(\bmx)$ 
of the function $f(\bmx)$ of $n$ unknowns $\bmx = (x_1, \ldots, x_n)^T$
at a sequence of given distinct $\{\bmx^{(k)}\}$, $k \geq 0$.
More specifically, given {\bf differences}
\[ \bms^{(k)} = \bmx^{(k+1)} - \bmx^{(k)} \;\;\mbox{and}\;\;
   \bmy^{(k)} = \bmg(\bmx^{(k+1)}) - \bmg(\bmx^{(k)})
\]
the package aims to find an approximation $\bmB$ to $\bmH(\bmx)$ for
which the secant conditions $\bmB \bms^{(k)} \approx \bmy^{(k)}$ hold for
a chosen set of values $k$.
The methods provided take advantage of the entries in the Hessian that
are known to be zero.

The package is particularly intended to allow gradient-based
optimization methods, that generate iterates 
$\bmx^{(k+1)} = \bmx^{(k)} + \bms^{(k)}$ based upon the values $\bmg( \bmx^{(k)})$
for $k \geq 0$, to build a suitable approximation to the Hessian 
$\bmH(\bmx^{(k+1)})$. This then gives the method an opportunity to 
accelerate the iteration using the Hessian approximation.

%%%%%%%%%%%%%%%%%%%%%% attributes %%%%%%%%%%%%%%%%%%%%%%%%

\galattributes
\galversions{\tt  \fullpackagename\_single, \fullpackagename\_double}.
\galuses 
{\tt GALAHAD\_SY\-M\-BOLS}, 
{\tt GALAHAD\_SP\-ECFILE} and
{\tt GALAHAD\_SPACE}.
\galdate August 2023.
\galorigin N. I. M. Gould, STFC-Rutherford Appleton Laboratory,
\gallanguage Fortran~2003. 

%%%%%%%%%%%%%%%%%%%%%% HOW TO USE %%%%%%%%%%%%%%%%%%%%%%%%

\galhowto

%\subsection{Calling sequences}

Access to the package requires a {\tt USE} statement such as

\medskip\noindent{\em Single precision version}

\hspace{8mm} {\tt USE \fullpackagename\_single}

\medskip\noindent{\em Double precision version}

\hspace{8mm} {\tt USE  \fullpackagename\_double}

\medskip

\noindent
The user is {\bf strongly advised} to use the double
precision version unless single precision corresponds to 8-byte arithmetic.

If it is required to use both modules at the same time, the derived types 
{\tt \packagename\_control\_type}, 
{\tt \packagename\_inform\_type},
{\tt \packagename\_data\_type}
and
{\tt NLPT\_userdata\_type},
(Section~\ref{galtypes})
and the subroutines
{\tt \packagename\_initialize}, 
{\tt \packagename\_\-analyse},
{\tt \packagename\_\-estimate},
{\tt \packagename\_terminate},
(Section~\ref{galarguments})
and 
{\tt \packagename\_read\_specfile}
(Section~\ref{galfeatures})
must be renamed on one of the {\tt USE} statements.

%%%%%%%%%%%%%%%%%%%%%% derived types %%%%%%%%%%%%%%%%%%%%%%%%

\galtypes
Four derived data types are accessible from the package.

%%%%%%%%%%% control type %%%%%%%%%%%

\subsubsection{The derived data type for holding control 
 parameters}\label{typecontrol}
The derived data type 
{\tt \packagename\_control\_type} 
is used to hold controlling data. Default values may be obtained by calling 
{\tt \packagename\_initialize}
(see Section~\ref{subinit}),
while components may also be changed by calling 
{\tt \fullpackagename\_read\-\_spec}
(see Section~\ref{readspec}). 
The components of 
{\tt \packagename\_control\_type} 
are:


\begin{description}

\itt{error} is a scalar variable of type default \integer, that holds the
stream number for error messages. Printing of error messages in 
{\tt \packagename\_analyse},
{\tt \packagename\_estimate} 
and {\tt \packagename\_terminate} 
is suppressed if {\tt error} $\leq 0$.
The default is {\tt error = 6}.

\ittf{out} is a scalar variable of type default \integer, that holds the
stream number for informational messages. Printing of informational messages in 
{\tt \packagename\_analyse} and {\tt \packagename\_estimate} 
is suppressed if {\tt out} $< 0$.
The default is {\tt out = 6}.

\itt{print\_level} is a scalar variable of type default \integer, that is used
to control the amount of informational output which is required. No 
informational output will occur if {\tt print\_level} $\leq 0$. If 
{\tt print\_level} $> 01$, details of any data errors encountered 
will be reported.
The default is {\tt print\_level = 0}.

\itt{approximation\_algorithm} is a scalar variable of type default \integer, 
that is used to select which approximation algorithm employed. This may be
\begin{enumerate}
\item 1. unsymmetric (Algorithm 2.1 in paper)
\item 2. symmetric (Algorithm 2.2 in paper)
\item 3. composite (Algorithm 2.3 in paper)
\item 4. composite 2 (Algorithm 2.4 in paper)
\item 5. cautious (Algorithm 2.5 in paper)
\end{enumerate}
Any value outside this range will be reset to the default;
the default is {\tt approximation\_algorithm = 4}.

\itt{dense\_linear\_solver} is a scalar variable of type default \integer, 
that specifies which (LAPACK) dense linear equation solver to use when finding
the values of entries in each row of $\bmB$. This may be
\begin{enumerate}
\item 1. Gaussian elimination
\item 2. QR factorization
\item 3. singular-value decomposition
\item 4. singular-value decomposition with divide-and-conquer
\end{enumerate}
Any value outside this range will be reset to the default;
the default is {\tt dense\_linear\_solver = 3}.

\itt{max\_sparse\_degree} is a scalar variable of type default \integer, 
that is used to specify the maximum sparse degree if a composite algorithm
({\tt \%approximation\_algorithm = 3, 4} is employed.
The default is {\tt max\_sparse\_degree = 50}.

\itt{extra\_differences} is a scalar variable of type default \integer, 
that is used to specify how may additional gradients (in addition to the number
output in {\tt inform\%differences\_needed} from {\tt \packagename\_\-analyse})
are available when calling {\tt \packagename\_\-estimate}.
The default is {\tt extra\_differences = 0}.

\itt{space\_critical} is a scalar variable of type default \logical, 
that must be set \true\ if space is critical when allocating arrays
and  \false\ otherwise. The package may run faster if 
{\tt space\_critical} is \false\ but at the possible expense of a larger
storage requirement. The default is {\tt space\_critical = .FALSE.}.

\itt{deallocate\_error\_fatal} is a scalar variable of type default \logical, 
that must be set \true\ if the user wishes to terminate execution if
a deallocation  fails, and \false\ if an attempt to continue
will be made. The default is {\tt deallocate\_error\_fatal = .FALSE.}.

\itt{prefix} is a scalar variable of type default \character\
and length 30, that may be used to provide a user-selected 
character string to preface every line of printed output. 
Specifically, each line of output will be prefaced by the string 
{\tt prefix(2:LEN(TRIM( prefix ))-1)},
thus ignoreing the first and last non-null components of the
supplied string. If the user does not want to preface lines by such
a string, they may use the default {\tt prefix = ""}.

\end{description}

%%%%%%%%%%% inform type %%%%%%%%%%%

\subsubsection{The derived data type for holding informational
 parameters}\label{typeinform}
The derived data type 
{\tt \packagename\_inform\_type} 
is used to hold parameters that give information about the progress and needs 
of the algorithm. The components of 
{\tt \packagename\_inform\_type} 
are:

\begin{description}
\itt{status} is a scalar variable of type default \integer, that gives the
exit status of the algorithm. 
See Section~\ref{galerrors} for details.

\itt{alloc\_status} is a scalar variable of type default \integer, that gives
the status of the last attempted array allocation or deallocation.
This will be 0 if {\tt status = 0}.

\itt{bad\_alloc} is a scalar variable of type default \character\
and length 80, that  gives the name of the last internal array 
for which there were allocation or deallocation errors.
This will be the null string if {\tt status = 0}. 

\itt{max\_degree} is a scalar variable of type default \integer, 
that holds the maximum degree in the adjacency graph.

\itt{differences\_needed} is a scalar variable of type default \integer, 
that holds the number of differences that will be needed (more may be helpful)
by {\tt \packagename\_\-estimate}. This value is computed by
{\tt \packagename\_\-analyse}.

\itt{max\_reduced\_degree} is a scalar variable of type default \integer, 
that holds the maximum reduced degree in the adjacency graph.

\itt{approximation\_algorithm\_used}  is a scalar variable of type 
default \integer, that specifies the actual approximation algorithm used
(see {\tt control\%approximation\_algorithm}).

\itt{bad\_row} is a scalar variable of type default \integer, that holds the 
index of the first row for which a failure occurred when forming its Hessian
values (or 0 if the data if no failures occurred).

\end{description}

%%%%%%%%%%% data type %%%%%%%%%%%

\subsubsection{The derived data type for holding problem data}\label{typedata}
The derived data type 
{\tt \packagename\_data\_type} 
is used to hold all the data for a particular problem,
or sequences of problems with the same structure, between calls of 
{\tt \packagename} procedures. 
This data should be preserved, untouched from the initial call to 
{\tt \packagename\_initialize}
to the final call to
{\tt \packagename\_terminate}.

%%%%%%%%%%%%%%%%%%%%%% argument lists %%%%%%%%%%%%%%%%%%%%%%%%

\galarguments
There are four procedures for user calls
(see Section~\ref{galfeatures} for further features): 

\begin{enumerate}
\item The subroutine 
      {\tt \packagename\_initialize} 
      is used to set default values, and initialize private data, 
      before solving one or more problems with the
      same sparsity and bound structure.
\item The subroutine 
      {\tt \packagename\_analyse} 
      is called to analyze the sparsity pattern of the Hessian
      and to generate information that will be used when estimating 
      its values.
\item The subroutine 
      {\tt \packagename\_estimate} 
      is called to estimate the Hessian by component-wise secant 
      approximation. This must be preceded by a call to 
      {\tt \packagename\_analyse}.
\item The subroutine 
      {\tt \packagename\_terminate} 
      is provided to allow the user to automatically deallocate array 
       components of the private data, allocated by 
       {\tt \packagename\_solve}, 
       at the end of the solution process. 
       It is important to do this if the data object is re-used for another 
       problem {\bf with a different structure}
       since {\tt \packagename\_initialize} cannot test for this situation, 
       and any existing associated targets will subsequently become unreachable.
\end{enumerate}
We use square brackets {\tt [ ]} to indicate \optional\ arguments.

%%%%%% initialization subroutine %%%%%%

\subsubsection{The initialization subroutine}\label{subinit}
 Default values are provided as follows:
\vspace*{1mm}

\hspace{8mm}
{\tt CALL \packagename\_initialize( data, control, inform )}

\vspace*{-3mm}
\begin{description}

\itt{data} is a scalar \intentinout\ argument of type 
{\tt \packagename\_data\_type}
(see Section~\ref{typedata}). It is used to hold data about the problem being 
solved. 

\itt{control} is a scalar \intentout\ argument of type 
{\tt \packagename\_control\_type}
(see Section~\ref{typecontrol}). 
On exit, {\tt control} contains default values for the components as
described in Section~\ref{typecontrol}.
These values should only be changed after calling 
{\tt \packagename\_initialize}.

\itt{inform} is a scalar \intentout\ argument of type 
{\tt \packagename\_inform\_type}
(see Section~\ref{typeinform}). A successful call to
{\tt \packagename\_initialize}
is indicated when the  component {\tt status} has the value 0. 
For other return values of {\tt status}, see Section~\ref{galerrors}.

\end{description}

%%%%%%%%% analysis subroutine %%%%%%

\subsubsection{The analysis subroutine}
The analysis phase, in which the given sparsity pattern of the Hessian
is used to generate information that will be used when estimating 
its values, is called as follows:

\vspace*{1mm}

\hspace{8mm}
{\tt CALL \packagename\_analyse( n, nz, ROW, COL, data, control, inform )}

\vspace*{-2mm}
\begin{description}
\ittf{n} 
is a scalar \intentin\ scalar argument of type default \integer, that must be 
set to $n$ the dimension of the Hessian matrix, i.e. the number of
variables in the function $f$. 
\restrictions {\tt n} $>$ {\tt 0}.

\ittf{nz} 
is a scalar \intentin\ scalar argument of type default \integer, that must be
set to the number of nonzero entries on and above the diagonal of the Hessian
matrix.
\restrictions {\tt nz} $\ge$ {\tt 0}.

\ittf{ROW} and {\tt COL} are a scalar \intentin\ rank-one array arguments 
of type  default \integer\ and dimension {\tt nz}, that are used to describe 
the sparsity structure of the Hessian matrix, $\bmH(\bmx)$. 
They must be set so that
{\tt ROW(}$i${\tt)} and {\tt COL(}$i${\tt)}, $i = 1, \ldots,$ {\tt nz},
contains the row and column indices of the nonzero elements of the 
{\bf upper triangular part, including the diagonal,} of the Hessian matrix.
The entries may appear in any order.
\restrictions 
$1 \leq$ {\tt ROW(}$j${\tt) $\leq$ COL(}$j${\tt)} $\leq n$, 
$j = 1, \ldots,$ {\tt nz}.

\itt{data} is a scalar \intentinout\ argument of type 
{\tt \packagename\_data\_type}
(see Section~\ref{typedata}). It is used to hold data about the problem being 
solved. It must not have been altered {\bf by the user} since the last call to 
{\tt \packagename\_initialize}.

\itt{control} is a scalar \intentin\ argument of type 
{\tt \packagename\_control\_type}
(see Section~\ref{typecontrol}). Default values may be assigned by calling 
{\tt \packagename\_initialize} prior to the first call to 
{\tt \packagename\_analyse}. 

\itt{inform} is a scalar \intentinout\ argument of type 
{\tt \packagename\_inform\_type}
(see Section~\ref{typeinform}). 
A successful call to
{\tt \packagename\_analyse}
is indicated when the  component {\tt status} has the value 0. 
For other return values of {\tt status}, see Section~\ref{galerrors}.

\end{description}

%%%%%%%%% estimation subroutine %%%%%%

\subsubsection{The estimation subroutine}
The estimation phase, in which the nonzero entries of the Hessian
are found by component-wise secant approximation, is called as follows:
\vspace*{1mm}

\hspace{8mm}
{\tt CALL \packagename\_estimate( n, nz, ROW, COL, m\_available, S, ls1, ls2, \&}
\vspace*{-1mm}

\hspace{42mm}
{\tt Y, ly1, ly2, VAL, data, control, inform[, ORDER] )}

\vspace*{-2mm}
\begin{description}
\itt{n,} {\tt nz}, {\tt ROW} and {\tt COL} are  \intentin\ arguments 
exactly as described and input to {\tt \packagename\_analyse},
and must not have been changed in the interim.

\itt{m\_available} 
is a scalar \intentin\ scalar argument of type default \integer, that should 
be set to the number of differences provided; ideally this will
be as large as {\tt inform\%differences\_needed} as reported by
{\tt \packagename\_\-analyse}, but better still there should be a 
further {\tt control\%extra\_differences} to allow for unlikely singularities.

\ittf{S} is a scalar \intentin\ rank-two array argument of type 
default \realdp, and dimension ({\tt ls1, ls2}), that should be set on input
so that the $i$-th entry of the $k$-th difference $s_i^{(k)}$ lies in
{\tt S(}$i, k${\tt )}.  

\ittf{ls1} 
is a scalar \intentin\ scalar argument of type default \integer, that must be
set to the length of the leading dimension of {\tt S}, and must be at least
{\tt n}. 

\ittf{ls2} 
is a scalar \intentin\ scalar argument of type default \integer, that must be
set to the length of the trailing dimension of {\tt S}, and must be at least
{\tt m\_available}. 

\ittf{Y} is a scalar \intentin\ rank-two array argument of type 
default \realdp, and dimension ({\tt ly1, ly2}), that should be set on input
so that the $i$-th entry of the $k$-th difference $y_i^{(k)}$ lies in
{\tt Y(}$i, k${\tt )}.  

\ittf{ly1} 
is a scalar \intentin\ scalar argument of type default \integer, that must be
set to the length of the leading dimension of {\tt Y}, and must be at least
{\tt n}. 

\ittf{ly2} 
is a scalar \intentin\ scalar argument of type default \integer, that must be
set to the length of the trailing dimension of {\tt Y}, and must be at least
{\tt m\_available}. 

\ittf{VAL} is a scalar \intentout\ rank-one array argument of type 
default \realdp, and dimension {\tt nz}, that will be set on output
to the non-zeros of the Hessian approximation $\bmB$ in the
order defined by the list stored in {\tt ROW} and {\tt COL}.

\itt{data} is a scalar \intentinout\ argument of type 
{\tt \packagename\_data\_type}
(see Section~\ref{typedata}). It is used to hold data about the problem being 
solved. It must not have been altered {\bf by the user} since the last call to 
{\tt \packagename\_initialize}.

\itt{control} is a scalar \intentin\ argument of type 
{\tt \packagename\_control\_type}
(see Section~\ref{typecontrol}). Default values may be assigned by calling 
{\tt \packagename\_initialize} prior to the first call to 
{\tt \packagename\_analyse}. 

\itt{inform} is a scalar \intentinout\ argument of type 
{\tt \packagename\_inform\_type}
(see Section~\ref{typeinform}). 
A successful call to
{\tt \packagename\_analyse}
is indicated when the  component {\tt status} has the value 0. 
For other return values of {\tt status}, see Section~\ref{galerrors}.

\itt{ORDER} is an \optional\ scalar \intentin\ rank-one array argument 
of type default integer and dimension {\tt m\_available}, that can be set
to the preferred order of access of the differences stored 
in {\tt S} and {\tt Y}. The calculation of each row of the 
Hessian approximation $\bmB$ depends on the number of nonzeros in the row,
and {\tt ORDER} allows the user to specify the order in which the columns
of  {\tt S} and {\tt Y} are accessed to determine these row values. 
If {\tt ORDER} is \present\, the $i$-th accessed column will be
{\tt ORDER(}$i${\tt )}. Otherwise the columns will be accessed
in their natural order $i, i = 1, \ldots ,$ {\tt m\_available}.

\end{description}

%%%%%%% termination subroutine %%%%%%

\subsubsection{The  termination subroutine}
All previously allocated arrays are deallocated as follows:
\vspace*{1mm}

\hspace{8mm}
{\tt CALL \packagename\_terminate( data, control, inform )}

\vspace*{-1mm}
\begin{description}

\itt{data} is a scalar \intentinout\ argument of type 
{\tt \packagename\_data\_type} 
exactly as for
{\tt \packagename\_solve},
which must not have been altered {\bf by the user} since the last call to 
{\tt \packagename\_initialize}.
On exit, array components will have been deallocated.

\itt{control} is a scalar \intentin\ argument of type 
{\tt \packagename\_control\_type}
exactly as for
{\tt \packagename\_solve}.

\itt{inform} is a scalar \intentout\ argument of type
{\tt \packagename\_inform\_type}
exactly as for
{\tt \packagename\_solve}.
Only the component {\tt status} will be set on exit, and a 
successful call to 
{\tt \packagename\_terminate}
is indicated when this  component {\tt status} has the value 0. 
For other return values of {\tt status}, see Section~\ref{galerrors}.

\end{description}

%%%%%%%%%%%%%%%%%%%%%% Warning and error messages %%%%%%%%%%%%%%%%%%%%%%%%

\galerrors
A negative value of {\tt inform\%status} on exit from 
{\tt \packagename\_analyse},
{\tt \packagename\_estimate}
or 
{\tt \packagename\_terminate}
indicates that an error has occurred. No further calls should be made
until the error has been corrected. Possible values are:

\begin{description}

\itt{\galerrallocate.} An allocation error occurred. 
A message indicating the offending
array is written on unit {\tt control\%error}, and the returned allocation 
status and a string containing the name of the offending array
are held in {\tt inform\%alloc\_\-status}
and {\tt inform\%bad\_alloc}, respectively.

\itt{\galerrdeallocate.} A deallocation error occurred. 
A message indicating the offending 
array is written on unit {\tt control\%error} and the returned allocation 
status and a string containing the name of the offending array
are held in {\tt inform\%alloc\_\-status}
and {\tt inform\%bad\_alloc}, respectively.

\itt{\galerrrestrictions.} 
One or more of the restrictions
{\tt n} $>$ {\tt 0},  {\tt nz} $\ge$ {\tt 0}, 
{\tt 1} $\le$ {\tt ROW(}$j${\tt )} $\le$ {\tt COL(}$j${\tt )} $\le$ {\tt n}, 
$j = 1, \ldots,$ {\tt nz},
has been violated.

\itt{\galerrfactorization.} 
The LAPACK dense linear equation solver used to find the values of
the rows of $\bmB$ has failed.

\itt{\galerrcallorder.}  {\tt \packagename\_estimate} has been called
before {\tt \packagename\_analyse}. 

\end{description}

%%%%%%%%%%%%%%%%%%%%%% Further features %%%%%%%%%%%%%%%%%%%%%%%%

\galfeatures
\noindent In this section, we describe an alternative means of setting 
control parameters, that is components of the variable {\tt control} of type
{\tt \packagename\_control\_type}
(see Section~\ref{typecontrol}), 
by reading an appropriate data specification file using the
subroutine {\tt \packagename\_read\_specfile}. This facility
is useful as it allows a user to change  {\tt \packagename} control parameters 
without editing and recompiling programs that call {\tt \packagename}.

A specification file, or specfile, is a data file containing a number of 
"specification commands". Each command occurs on a separate line, 
and comprises a "keyword", 
which is a string (in a close-to-natural language) used to identify a 
control parameter, and 
an (optional) "value", which defines the value to be assigned to the given
control parameter. All keywords and values are case insensitive, 
keywords may be preceded by one or more blanks but
values must not contain blanks, and
each value must be separated from its keyword by at least one blank.
Values must not contain more than 30 characters, and 
each line of the specfile is limited to 80 characters,
including the blanks separating keyword and value.

The portion of the specification file used by 
{\tt \packagename\_read\_specfile}
must start
with a "{\tt BEGIN \packagename}" command and end with an 
"{\tt END}" command.  The syntax of the specfile is thus defined as follows:
\begin{verbatim}
  ( .. lines ignored by SHA_read_specfile .. )
    BEGIN SHA
       keyword    value
       .......    .....
       keyword    value
    END 
  ( .. lines ignored by SHA_read_specfile .. )
\end{verbatim}
where keyword and value are two strings separated by (at least) one blank.
The ``{\tt BEGIN \packagename}'' and ``{\tt END}'' delimiter command lines 
may contain additional (trailing) strings so long as such strings are 
separated by one or more blanks, so that lines such as
\begin{verbatim}
    BEGIN SHA SPECIFICATION
\end{verbatim}
and
\begin{verbatim}
    END SHA SPECIFICATION
\end{verbatim}
are acceptable. Furthermore, 
between the
``{\tt BEGIN \packagename}'' and ``{\tt END}'' delimiters,
specification commands may occur in any order.  Blank lines and
lines whose first non-blank character is {\tt !} or {\tt *} are ignored. 
The content 
of a line after a {\tt !} or {\tt *} character is also 
ignored (as is the {\tt !} or {\tt *}
character itself). This provides an easy manner to "comment out" some 
specification commands, or to comment specific values 
of certain control parameters.  

The value of a control parameters may be of three different types, namely
integer, logical or real.
Integer and real values may be expressed in any relevant Fortran integer and
floating-point formats (respectively). Permitted values for logical
parameters are "{\tt ON}", "{\tt TRUE}", "{\tt .TRUE.}", "{\tt T}", 
"{\tt YES}", "{\tt Y}", or "{\tt OFF}", "{\tt NO}",
"{\tt N}", "{\tt FALSE}", "{\tt .FALSE.}" and "{\tt F}". 
Empty values are also allowed for 
logical control parameters, and are interpreted as "{\tt TRUE}".  

The specification file must be open for 
input when {\tt \packagename\_read\_specfile}
is called, and the associated device number 
passed to the routine in device (see below). 
Note that the corresponding 
file is {\tt REWIND}ed, which makes it possible to combine the specifications 
for more than one program/routine.  For the same reason, the file is not
closed by {\tt \packagename\_read\_specfile}.

\subsubsection{To read control parameters from a specification file}
\label{readspec}

Control parameters may be read from a file as follows:
\hskip0.5in 
\def\baselinestretch{0.8} {\tt \begin{verbatim}
     CALL SHA_read_specfile( control, device )
\end{verbatim}
}
\def\baselinestretch{1.0}

\begin{description}
\itt{control} is a scalar \intentinout argument of type 
{\tt \packagename\_control\_type}
(see Section~\ref{typecontrol}). 
Default values should have already been set, perhaps by calling 
{\tt \packagename\_initialize}.
On exit, individual components of {\tt control} may have been changed
according to the commands found in the specfile. Specfile commands and 
the component (see Section~\ref{typecontrol}) of {\tt control} 
that each affects are given in Table~\ref{specfile}.

\bctable{|l|l|l|} 
\hline
  command & component of {\tt control} & value type \\ 
\hline
  {\tt error-printout-device} & {\tt \%error} & integer \\
  {\tt printout-device} & {\tt \%out} & integer \\
  {\tt print-level} & {\tt \%print\_level} & integer \\
  {\tt approximation-algorithm} & {\tt \%approximation\_algorithm} & integer \\
  {\tt dense-linear-solver} & {\tt \%dense\_linear\_solver} & integer \\
  {\tt maximum-degree-considered-sparse} & {\tt \%max\_sparse\_degree} & integer \\
  {\tt extra-differences} & {\tt \%extra\_differences} & integer \\
  {\tt space-critical}   & {\tt \%space\_critical} & logical \\
  {\tt deallocate-error-fatal}   & {\tt \%deallocate\_error\_fatal} & logical \\
  {\tt output-line-prefix} & {\tt \%prefix} & character \\
\hline

\ectable{\label{specfile}Specfile commands and associated 
components of {\tt control}.}

\itt{device} is a scalar \intentin argument of type default \integer,
that must be set to the unit number on which the specfile
has been opened. If {\tt device} is not open, {\tt control} will
not be altered and execution will continue, but an error message
will be printed on unit {\tt control\%error}.

\end{description}

%%%%%%%%%%%%%%%%%%%%%% Information printed %%%%%%%%%%%%%%%%%%%%%%%%

\galinfo
If {\tt control\%print\_level} is positive, information about 
errors encountered will be printed on unit {\tt control\-\%out}.

%%%%%%%%%%%%%%%%%%%%%% GENERAL INFORMATION %%%%%%%%%%%%%%%%%%%%%%%%

\galgeneral

\galcommon None.
\galworkspace Provided automatically by the module.
\galroutines None. 
\galmodules {\tt \packagename\_solve} calls the \galahad\ packages
{\tt GALAHAD\_SY\-M\-BOLS}, 
{\tt GALAHAD\_SPECFILE} and
{\tt GALAHAD\_NLPT}.
\galio Output is under control of the arguments
 {\tt control\%error}, {\tt control\%out} and {\tt control\%print\_level}.
\galrestrictions 
$0 <$ {\tt n},  $0 \le$ {\tt nz},
{\tt 1} $\le$ {\tt ROW(}$j${\tt )} $\le$ {\tt COL(}$j${\tt )} $\le$ {\tt n}, 
$j = 1, \ldots,$ {\tt nz}.
\galportability Fortran~2003. 
The package is thread-safe.

%%%%%%%%%%%%%%%%%%%%%% METHOD %%%%%%%%%%%%%%%%%%%%%%%%

\galmethod
The package computes the entries in the each row of $\bmB$ one at a time.
The entries $b_{ij}$ in row $i$ may be chosen to
\eqn{ss}{\minin{b_{i,j}} \;\; \sum_{k \in {\cal I}_i} 
 \left[ \sum_{{\scriptscriptstyle \mbox{nonzeros}}\; j} b_{i,j} s_j^{(k)} - y_i^{(k)} \right]^2,}
where ${\cal I}_i$ is ideally chosen to be sufficiently large so that \req{ss}
has a unique minimizer. Since this requires that there are at least
as many $(\bms^{(k)}, \bmy^{(k)})$ pairs as the maximum number of nonzeros
in any row, this may be prohibitive in some cases. We might then be content 
with a minimum-norm (under-determined) least-squares solution. Or, we may
take advantage of the symmetry of the Hessian, and note that if we
have already found the values in row $j$, then the value $b_{i,j} = b_{j,i}$
in \req{ss} is known before we process row $i$. Thus by ordering the rows 
and exploiting symmetry we may reduce the numbers of unknowns in 
future unprocessed rows.

In the analysis phase, we order the rows by constructing the connectivity
graph---a graph comprising nodes $1$ to $n$ and edges connecting 
nodes $i$ and $j$ if $h_{i,j}$ is everywhere nonzero---of $\bmH(\bmx)$.
The nodes are ordered by increasing degree (that is, the number of edges
emanating from the node) using a bucket sort. The row chosen to be
ordered next corresponds to a node of minimum degree, the node
is removed from the graph, the degrees updated efficiently, and the
process repeated until all rows have been ordered. This often leads
to a significant reduction in the numbers of unknown values in each
row as it is processed in turn, but numerical rounding can lead to
inaccurate values in some cases. A useful remedy is to process all
rows for which there are sufficient $(\bms^{(k)}, \bmy^{(k)})$ as before,
and then process the remaining rows taking into account the symmetry.
That is, the rows and columns are rearranged so that the matrix
is in block form
\disp{\bmB = \mat{cc}{ \bmB_{11} & \bmB_{12} \\ \bmB^T_{12} & \bmB_{22}},}
the $( \bmB_{11} \;\; \bmB_{12})$ rows are processed without regard
for symmetry but give the $2,1$ block $\bmB^T_{12}$, and finally
the $2,2$ block $\bmB_{22}$ is processed either with the option of exploiting
symmetry. More details of the precise algorithms (Algorithms 2.1--2.5)
are given in the reference below. The linear least-squares problems \req{ss} 
themselves are solved by a choice of LAPACK packages.

\vspace*{1mm}

\galreference
\vspace*{1mm}

\noindent
The method is described in detail in
\vspace*{1mm}

\noindent

J.\ M.\ Fowkes, N.\ I.\ M.\ Gould and J.\ A.\ Scott,
Approximating large-scale Hessians using secant equations.
Technical Report TR-2023, Rutherford Appleton Laboratory.

%%%%%%%%%%%%%%%%%%%%%% EXAMPLE %%%%%%%%%%%%%%%%%%%%%%%%

\galexamples
Suppose we wish to estimate the Hessian matrix whose values at
a given $\bmx$ are
\disp{\bmH(\bmx) = \mat{ccccc}{ 1 & 2 & 3 & 4 & 5 \\
 2 & 6 & 0 & 0 & 0 \\ 3 & 0 & 7 & 0 & 0 \\
 4 & 0 & 0 & 8 & 0 \\ 5 & 0 & 0 & 0 & 9}}
and that we have (artificially) sampled the matrix via
$\bmy^{(k)} = \bmH(\bmx) \bms^{(k)}$ along random vectors
$\bms^{(k)}$ for $k = 1, \ldots, k_s$; a suitable value for $k_s$ is
returned by {\tt \packagename\_analyse}. Then we may recover $\bmH(\bmx)$
as follows:

{\tt \small
\VerbatimInput{\packageexample}
}
\noindent
The code produces the following output:
{\tt \small
\VerbatimInput{\packageresults}
}
\noindent

\end{document}

