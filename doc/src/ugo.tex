\documentclass{galahad}

% set the package name

\newcommand{\package}{ugo}
\newcommand{\packagename}{UGO}
\newcommand{\fullpackagename}{\libraryname\_\packagename}
\newcommand{\solver}{{\tt \fullpackagename\_solve}}

\begin{document}

\galheader

%%%%%%%%%%%%%%%%%%%%%% SUMMARY %%%%%%%%%%%%%%%%%%%%%%%%

\galsummary
This package aims to find the {\bf global minimizer of a univariate
twice-continuously differentiable function $f(x)$ of a single variable
over the finite interval $x^l \leq x \leq x^u$.} Function and derivative
values may be provided either via a subroutine call, or by a return
to the calling program. Second derivatives may be used to advantage if
they are available.

%%%%%%%%%%%%%%%%%%%%%% attributes %%%%%%%%%%%%%%%%%%%%%%%%

\galattributes
\galversions{\tt  \fullpackagename\_single, \fullpackagename\_double}.
\galuses
{\tt GALAHAD\_CLOCK},
{\tt GALAHAD\_SY\-M\-BOLS},
{\tt GALAHAD\_SPECFILE},
{\tt GALAHAD\_SPACE},
{\tt GALAHAD\_STRINGS},
and
{\tt GALAHAD\_USERDATA}.
\galdate May 2016.
\galorigin N. I. M. Gould, Rutherford Appleton Laboratory.
\gallanguage Fortran~95 + TR 15581 or Fortran~2003.

%%%%%%%%%%%%%%%%%%%%%% HOW TO USE %%%%%%%%%%%%%%%%%%%%%%%%

\galhowto

%\subsection{Calling sequences}

Access to the package requires a {\tt USE} statement such as

\medskip\noindent{\em Single precision version}

\hspace{8mm} {\tt USE \fullpackagename\_single}

\medskip\noindent{\em Double precision version}

\hspace{8mm} {\tt USE  \fullpackagename\_double}

\medskip

\noindent
If it is required to use both modules at the same time, the derived types
{\tt \packagename\_time\_\-type},
{\tt \packagename\_control\_type},
{\tt \packagename\_inform\_type},
{\tt \packagename\_data\_type}
and
(Section~\ref{galtypes})
and the subroutines
{\tt \packagename\_initialize},
{\tt \packagename\_\-solve},
{\tt \packagename\_terminate},
(Section~\ref{galarguments})
and
{\tt \packagename\_read\_specfile}
(Section~\ref{galfeatures})
must be renamed on one of the {\tt USE} statements.

%%%%%%%%%%%%%%%%%%%%%% derived types %%%%%%%%%%%%%%%%%%%%%%%%

\galtypes
Four derived data types are accessible from the package.

%%%%%%%%%%% control type %%%%%%%%%%%

\subsubsection{The derived data type for holding control
 parameters}\label{typecontrol}
The derived data type
{\tt \packagename\_control\_type}
is used to hold controlling data. Default values may be obtained by calling
{\tt \packagename\_initialize}
(see Section~\ref{subinit}),
while components may also be changed by calling
{\tt \fullpackagename\_read\-\_spec}
(see Section~\ref{readspec}).
The components of
{\tt \packagename\_control\_type}
are:

\begin{description}

\itt{error} is a scalar variable of type default \integer, that holds the
stream number for error messages. Printing of error messages in
{\tt \packagename\_solve} and {\tt \packagename\_terminate}
is suppressed if {\tt error} $\leq 0$.
The default is {\tt error = 6}.

\ittf{out} is a scalar variable of type default \integer, that holds the
stream number for informational messages. Printing of informational messages in
{\tt \packagename\_solve} is suppressed if {\tt out} $< 0$.
The default is {\tt out = 6}.

\itt{print\_level} is a scalar variable of type default \integer, that is used
to control the amount of informational output which is required. No
informational output will occur if {\tt print\_level} $\leq 0$. If
{\tt print\_level} $= 1$, a single line of output will be produced for each
iteration of the process. If {\tt print\_level} $\geq 2$, this output will be
increased to provide significant detail of each iteration.
The default is {\tt print\_level = 0}.

\itt{start\_print} is a scalar variable of type default \integer, that specifies
the first iteration for which printing will occur in {\tt \packagename\_solve}.
If {\tt start\_print} is negative, printing will occur from the outset.
The default is {\tt start\_print = -1}.

\itt{stop\_print} is a scalar variable of type default \integer, that specifies
the last iteration for which printing will occur in  {\tt \packagename\_solve}.
If {\tt stop\_print} is negative, printing will occur once it has been
started by {\tt start\_print}.
The default is {\tt stop\_print = -1}.

\itt{print\_gap} is a scalar variable of type default \integer.
Once printing has been started, output will occur once every
{\tt print\_gap} iterations. If {\tt print\_gap} is no larger than 1,
printing will be permitted on every iteration.
The default is {\tt print\_gap = 1}.

\itt{maxit} is a scalar variable of type default \integer, that holds the
maximum number of iterations which will be allowed in {\tt \packagename\_solve}.
The default is {\tt maxit = 1000}.

\itt{initial\_points} is a scalar variable of type default \integer, that
gives the number of initial (uniformly-spaced) evaluation points; if
{\tt initial\_points} < 2, it will be reset to 2.
The default is {\tt initial\_points = 2}.

\itt{storage\_increment} is a scalar variable of type default \integer, that
specifies the incremement size of additional workspace storage that will be
allocated as needed; if {\tt storage\_increment} < 1000, it will be reset
to 1000.
The default is {\tt storage\_increment = 1000}.

\itt{buffer} is a scalar variable of type default \integer, that specifies
the unit number for any out-of-core i/o when expanding workspace arrays.
The default is {\tt buffer = 70}.

\itt{lipschitz\_estimate\_used} is a scalar variable of type default \integer,
that specifies what sort of Lipschitz constant estimate will be used.
Possible values are
\begin{description}
\item 1. the  global contant is provided in {\tt global\_lipschitz\_constant}.
\item 2. the global contant is estimated.
\item 3. local constants are estimated.
\end{description}
The default is {\tt lipschitz\_estimate\_used = 3}.

\itt{next\_interval\_selection} is a scalar variable of type default \integer,
that specifies how the next interval for examination is chosen. Possible
values are
\begin{description}
\item 1. traditional, based on the best overall predicted value
\item 2. a local improvement algorithm is used (not currently implemented,
  and defaults to 1)
\end{description}
The default is {\tt next\_interval\_selection = 1}.

\itt{refine\_with\_newton} is a scalar variable of type default \integer, that
specifies the maximum number of Newton steps that may be used in the
vacinity of a global  minimizer to try to improve the estimate.
The default is {\tt refine\_with\_newton = 5}.

\itt{alive\_unit} is a scalar variable of type default \integer.
If {\tt alive\_unit} $>$ 0, a temporary file named {\tt alive\_file} (see below)
will be created on stream number {\tt alive\_unit} on initial entry to
\solver, and execution of \solver\ will continue so
long as this file continues to exist. Thus, a user may terminate execution
simply by removing the temporary file from this unit.
If {\tt alive\_unit} $\leq$ 0, no temporary file will be created, and
execution cannot be terminated in this way.
The default is {\tt alive\_unit} $=$ 40.

\itt{stop\_length} is a scalar variable of type default \realdp, that is used to
assess convergence. The method will stop if all sub-intervals that could
cntain the global minimizer are smaller than {\tt stop\_length} in length.
The default is {\tt stop\_length =} $10^{-5}$.

\itt{small\_g\_for\_newton} is a scalar variable of type default \realdp,
that is used to assess when to use a Newton correction. If the absolute value
of the gradient (first derivative) of $f$ is smaller than
{\tt small\_g\_for\_newton}, the next evaluation point may be at a
Newton estimate of a local minimizer.
The default is {\tt small\_g\_for\_newton =} $10^{-3}$.

\itt{small\_g} is a scalar variable of type default \realdp, that is used to
assess when no Newton search is necessary. This will be the case if
the absolute value of the gradient at the end of the interval search is
smaller than {\tt small\_g}.
The default is {\tt = small\_g} $10^{-6}$.

\itt{obj\_sufficient} is a scalar variable of type default \realdp,
that may be used to stop the iteration if a value of $x$ is found for which
the objective function is sufficiently small. Specifically, the search
will stop if the objective function is smaller than {\tt obj\_sufficient}.
The default is {\tt obj\_sufficient =} $-u^{-2}$,
where $u$ is {\tt EPSILON(1.0)} ({\tt EPSILON(1.0D0)} in
{\tt \fullpackagename\_double}).

\itt{global\_lipschitz\_constant} is a scalar variable of type default \realdp,
that is used to specify the global Lipschitz constant for the gradient;
a negative value indicates that the value is unknown.
The default is {\tt global\_lipschitz\_constant =} -1.0.

\itt{reliability\_parameter} is a scalar variable of type default \realdp,
that is used to boost insufficiently large estimates of the Lipschitz constant
if necessary. If {\tt reliability\_parameter} is not positive, it will be
reset to {\tt 1.2} when second derivatives are provided
(see {\tt second\_derivatives\_available} below, and {\tt 2.0} if they
are not.
The default is {\tt reliability\_parameter =} -1.0.

\itt{lipschitz\_lower\_bound} is a scalar variable of type default \realdp,
that provides a lower bound on the Lipschitz constant for the gradient.
This must be non-negative (and not zero unless the function is constant).
The default is {\tt lipschitz\_lower\_bound =} $10^{-8}$.

\itt{cpu\_time\_limit} is a scalar variable of type default \realdp,
that is used to specify the maximum permitted CPU time. Any negative
value indicates no limit will be imposed. The default is
{\tt cpu\_time\_limit = - 1.0}.

\itt{clock\_time\_limit} is a scalar variable of type default \realdp,
that is used to specify the maximum permitted elapsed system clock time.
Any negative value indicates no limit will be imposed. The default is
{\tt clock\_time\_limit = - 1.0}.

\itt{second\_derivatives\_available} is a scalar variable of type default 
\logical, that must be set \true\ when second derivatives of $f(x)$ 
are available, and  \false\ otherwise. The package is generally more effective 
if second derivatives are available.
The default is {\tt second\_derivatives\_available = .TRUE.}.

\itt{space\_critical} is a scalar variable of type default \logical,
that must be set \true\ if space is critical when allocating arrays
and  \false\ otherwise. The package may run faster if
{\tt space\_critical} is \false\ but at the possible expense of a larger
storage requirement. The default is {\tt space\_critical = .FALSE.}.

\itt{deallocate\_error\_fatal} is a scalar variable of type default \logical,
that must be set \true\ if the user wishes to terminate execution if
a deallocation  fails, and \false\ if an attempt to continue
will be made. The default is {\tt deallocate\_error\_fatal = .FALSE.}.

\itt{alive\_file} is a scalar variable of type default \character\ and length
30, that gives the name of the temporary file whose removal from stream number
{\tt alive\_unit} terminates execution of \solver.
The default is {\tt alive\_unit} $=$ {\tt ALIVE.d}.

\itt{prefix} is a scalar variable of type default \character\
and length 30, that may be used to provide a user-selected
character string to preface every line of printed output.
Specifically, each line of output will be prefaced by the string
{\tt prefix(2:LEN(TRIM( prefix ))-1)},
thus ignoreing the first and last non-null components of the
supplied string. If the user does not want to preface lines by such
a string, they may use the default {\tt prefix = ""}.

\end{description}

%%%%%%%%%%% time type %%%%%%%%%%%

\subsubsection{The derived data type for holding timing
 information}\label{typetime}
The derived data type
{\tt \packagename\_time\_type}
is used to hold elapsed CPU and system clock times for the calculation.
The components of
{\tt \packagename\_time\_type}
are:
\begin{description}
\itt{total} is a scalar variable of type default \real, that gives
 the CPU total time spent in the package.

\itt{clock\_total} is a scalar variable of type default \real, that gives
 the total elapsed system clock time spent in the package.

\end{description}

%%%%%%%%%%% inform type %%%%%%%%%%%

\subsubsection{The derived data type for holding informational
 parameters}\label{typeinform}
The derived data type
{\tt \packagename\_inform\_type}
is used to hold parameters that give information about the progress and needs
of the algorithm. The components of
{\tt \packagename\_inform\_type}
are:

\begin{description}
\itt{status} is a scalar variable of type default \integer, that gives the
exit status of the algorithm.
See Sections~\ref{reverse} and \ref{galerrors} for details.

\itt{alloc\_status} is a scalar variable of type default \integer, that gives
the status of the last attempted array allocation or deallocation.
This will be 0 if {\tt status = 0}.

\itt{bad\_alloc} is a scalar variable of type default \character\
and length 80, that  gives the name of the last internal array
for which there were allocation or deallocation errors.
This will be the null string if {\tt status = 0}.

\ittf{iter} is a scalar variable of type default \integer, that holds the
number of iterations performed.

\itt{f\_eval} is a scalar variable of type default \integer, that gives the
total number of evaluations of the objective function.

\itt{g\_eval} is a scalar variable of type default \integer, that gives the
total number of evaluations of the first derivative of the objective function.

\itt{h\_eval} is a scalar variable of type default \integer, that gives the
total number of evaluations of the second derivative of the objective function.

\itt{dx\_best} is a scalar variable of type default \integer,
that gives the length of the largest remaining search interval.

\ittf{time} is a scalar variable of type {\tt \packagename\_time\_type}
whose components are used to hold elapsed elapsed CPU and system clock
times for the various parts of the calculation (see Section~\ref{typetime}).

\end{description}

%%%%%%%%%%% data type %%%%%%%%%%%

\subsubsection{The derived data type for holding problem data}\label{typedata}
The derived data type
{\tt \packagename\_data\_type}
is used to hold all the data for a particular problem,
or sequences of problems with the same structure, between calls of
{\tt \packagename} procedures.
This data should be preserved, untouched (except as directed on
return from \solver\ with positive values of {\tt inform\%status}, see
Section~\ref{reverse}),
from the initial call to
{\tt \packagename\_initialize}
to the final call to
{\tt \packagename\_terminate}.

%%%%%%%%%%% userdata type %%%%%%%%%%%

\subsubsection{The derived data type for holding user data}\label{typeuserdata}
The derived data type {\tt GALAHAD\_user\-data\_type}
is available from the package {\tt GALAHAD\_user\-data}
to allow the user to pass data to and from user-supplied
subroutines for function and derivative calculations (see Section~\ref{fdv}).
Components of variables of type {\tt GALAHAD\_user\-data\_type} may be 
allocated as necessary. The following components are available:

\begin{description}
\itt{integer} is a rank-one allocatable array of type default \integer.
\ittf{real} is a rank-one allocatable array of type default  \realdp
\itt{complex} is a rank-one allocatable array of type default \complexdp.
\itt{character} is a rank-one allocatable array of type default \character.
\itt{logical} is a rank-one allocatable array of type default \logical.
\itt{integer\_pointer} is a rank-one pointer array of type default \integer.
\itt{real\_pointer} is a rank-one pointer array of type default  \realdp
\itt{complex\_pointer} is a rank-one pointer array of type default \complexdp.
\itt{character\_pointer} is a rank-one pointer array of type default \character.
\itt{logical\_pointer} is a rank-one pointer array of type default \logical.
\end{description}


%%%%%%%%%%%%%%%%%%%%%% argument lists %%%%%%%%%%%%%%%%%%%%%%%%

\galarguments
There are three procedures for user calls
(see Section~\ref{galfeatures} for further features):

\begin{enumerate}
\item The subroutine
      {\tt \packagename\_initialize}
      is used to set default values, and initialize private data,
      before solving one or more problems with the
      same sparsity and bound structure.
\item The subroutine
      {\tt \packagename\_solve}
      is called to solve the problem.
\item The subroutine
      {\tt \packagename\_terminate}
      is provided to allow the user to automatically deallocate array
      components of the private data, allocated by
      {\tt \packagename\_solve},
      at the end of the solution process.
\end{enumerate}
We use square brackets {\tt [ ]} to indicate \optional\ arguments.

%%%%%% initialization subroutine %%%%%%

\subsubsection{The initialization subroutine}\label{subinit}
 Default values are provided as follows:
\vspace*{1mm}

\hspace{8mm}
{\tt CALL \packagename\_initialize( data, control, inform )}

\vspace*{-2mm}
\begin{description}

\itt{data} is a scalar \intentinout\ argument of type
{\tt \packagename\_data\_type}
(see Section~\ref{typedata}). It is used to hold data about the problem being
solved.

\itt{control} is a scalar \intentout\ argument of type
{\tt \packagename\_control\_type}
(see Section~\ref{typecontrol}).
On exit, {\tt control} contains default values for the components as
described in Section~\ref{typecontrol}.
These values should only be changed after calling
{\tt \packagename\_initialize}.

\itt{inform} is a scalar \intentout\ argument of type
{\tt \packagename\_inform\_type}
(see Section~\ref{typeinform}). A successful call to
{\tt \packagename\_initialize}
is indicated when the  component {\tt status} has the value 0.
For other return values of {\tt status}, see Section~\ref{galerrors}.

\end{description}

%%%%%%%%% main solution subroutine %%%%%%

\subsubsection{The minimization subroutine}
The minimization algorithm is called as follows:
\vspace*{1mm}

\hspace{8mm}
{\tt CALL \packagename\_solve( x\_l, x\_u, x, f, g, h, control, inform, data, \&
\vspace*{-1mm}

\hspace{37mm}
                        userdata[, eval\_FGH] )}
\vspace*{-2mm}
\begin{description}
\itt{x\_l} and {\tt x\_u} are scalar \intentin\ variables
of type default \realdp that must be set to the lower and upper
interval bounds $x^l$ and $x^u$, respectively.

\ittf{x} is a scalar \intentinout\ variable of type default \realdp that holds
the next value of $x$ at which the user is required to evaluate the
objective (and its derivatives) when {\tt inform\%status > 0},
or the value of the approximate global minimizer
when {\tt inform\%status = 0}.
{\tt x} need not be set on initial ({\tt inform\%status = 1}) entry.

\ittf{f} is a scalar \intentinout\ variable of type default \realdp that must
be set by the user to hold the value of the objective $f(x)$ at $x$ given
in {\tt x} whenever {\tt \packagename\_solve} returns with
{\tt inform\%status > 0}.
If {\tt inform\%status = 0}, {\tt f} will contain the value of the
approximate global minimum $f(x)$ at the approximate minimizer $x$
given in {\tt x}.
{\tt f} need not be set on initial ({\tt inform\%status = 1}) entry.

\ittf{g} is a scalar \intentinout\ variable of type default \realdp that must
be set by the user to hold the value of the first derivative $f^{\prime}(x)$
of $f(x)$ at $x$ given in {\tt x} whenever {\tt \packagename\_solve} returns
with {\tt inform\%status > 0}. If {\tt inform\%status = 0}, {\tt g}
will contain the value of the gradient of
$f(x)$ at the approximate minimizer $x$ given in {\tt x}.
{\tt g} need not be set on initial ({\tt inform\%status = 1}) entry.

\ittf{h} is a scalar \intentinout\ variable of type default \realdp that must
be set by the user to hold the value of the second derivative $f^{\prime\prime}(x)$
of $f(x)$ at $x$ given in {\tt x} whenever {\tt \packagename\_solve} returns
with {\tt inform\%status = 4}
and the user has set {\tt control\%second\_derivatives\_available = .TRUE.}
If {\tt inform\%status = 0}, {\tt h}
will contain the value of the second derivative of
$f(x)$ at the approximate minimizer $x$ given in {\tt x} when
second derivatives are available.
{\tt h} need not be set on initial ({\tt inform\%status = 1}) entry.

\itt{control} is a scalar \intentin\ argument of type
{\tt \packagename\_control\_type}
(see Section~\ref{typecontrol}). Default values may be assigned by calling
{\tt \packagename\_initialize} prior to the first call to
{\tt \packagename\_solve}.

\itt{inform} is a scalar \intentinout\ argument of type
{\tt \packagename\_inform\_type}
(see Section~\ref{typeinform}).
On initial entry, the  component {\tt status} must be set to the value 1.
Other entries need note be set.
A successful call to
{\tt \packagename\_solve}
is indicated when the  component {\tt status} has the value 0.
For other return values of {\tt status}, see Sections~\ref{reverse} and
\ref{galerrors}.

\itt{data} is a scalar \intentinout\ argument of type
{\tt \packagename\_data\_type}
(see Section~\ref{typedata}). It is used to hold data about the problem being
solved. With the possible exceptions of the components
{\tt eval\_status} and {\tt U} (see Section~\ref{reverse}),
it must not have been altered {\bf by the user} since the last call to
{\tt \packagename\_initialize}.

\itt{userdata} is a scalar \intentinout\ argument of type
{\tt GALAHAD\_userdata\_type} whose components may be used
to communicate user-supplied data to and from the
\optional\ subroutine {\tt eval\_FGH}
(see Section~\ref{typeuserdata}).

\itt{eval\_FGH} is an \optional\
user-supplied subroutine whose purpose is to evaluate the value of the
objective function $f(x)$ and its derivatives at a given vector $x$.
See Section~\ref{fghfv} for details. If {\tt eval\_FGH} is present,
it must be declared {\tt EXTERNAL} in the calling program.
If {\tt eval\_FGH} is absent, \solver\ will use reverse communication to
obtain objective function and derivative values (see Section~\ref{reverse}).

\end{description}

%%%%%%% termination subroutine %%%%%%

\subsubsection{The  termination subroutine}
All previously allocated arrays are deallocated as follows:
\vspace*{1mm}

\hspace{8mm}
{\tt CALL \packagename\_terminate( data, control, inform )}

\vspace*{-1mm}
\begin{description}

\itt{data} is a scalar \intentinout\ argument of type
{\tt \packagename\_data\_type}
exactly as for
{\tt \packagename\_solve},
which must not have been altered {\bf by the user} since the last call to
{\tt \packagename\_initialize}.
On exit, array components will have been deallocated.

\itt{control} is a scalar \intentin\ argument of type
{\tt \packagename\_control\_type}
exactly as for
{\tt \packagename\_solve}.

\itt{inform} is a scalar \intentout\ argument of type
{\tt \packagename\_inform\_type}
exactly as for
{\tt \packagename\_solve}.
Only the component {\tt status} will be set on exit, and a
successful call to
{\tt \packagename\_terminate}
is indicated when this  component {\tt status} has the value 0.
For other return values of {\tt status}, see Section~\ref{galerrors}.

\end{description}

%%%%%%%%%%%%%%%%%%%% Function and derivative values %%%%%%%%%%%%%%%%%%%%%

\subsection{Function and derivative values\label{fdv}}

%%%%%%%%%%%%%%%% Objective function and derivatives values %%%%%%%%%%%%%%%%%%%

\subsubsection{The objective function value via internal evaluation\label{fghfv}}

If the argument {\tt eval\_FGH} is present when calling \solver, the
user is expected to provide a subroutine of that name to evaluate the
value of the objective function $f(x)$ and its first and (possibly) 
second derivatives, $f^{\prime}(x)$ and $f^{\prime\prime}(x)$.

The routine must be specified as

\def\baselinestretch{0.8}
{\tt
\begin{verbatim}
      SUBROUTINE eval_FGH( status, x, userdata, f, g, h )
\end{verbatim}
}
\def\baselinestretch{1.0}
\noindent whose arguments are as follows:

\begin{description}
\itt{status} is a scalar \intentout\ argument of type default \integer,
that should be set to 0 if the routine has been able to evaluate
the objective function and its derivatives,
and to a non-zero value if the evaluation has not been possible.

\ittf{x} is a scalar \intentin\ array argument of type default \realdp\
whose components contain the value $x$.

\itt{userdata} is a scalar \intentinout\ argument of type
{\tt GALAHAD\_userdata\_type} whose components may be used
to communicate user-supplied data to and from the
subroutine (see Section~\ref{typeuserdata}).

\ittf{f} is a scalar \intentout\ argument of type default \realdp,
that should be set to the value of the objective function $f(x)$
evaluated at the vector $x$ input in {\tt x}.

\ittf{g} is a scalar \intentout\ argument of type default \realdp,
that should be set to the value of the first derivative
$f^{\prime}(x)$ of the objective function $f(x)$
evaluated at the vector $x$ input in {\tt x}.

\ittf{h} is a scalar \intentout\ argument of type default \realdp,
that should be set to the value of the second derivative
$f^{\prime\prime}(x)$ of the objective function $f(x)$
evaluated at the vector $x$ input in {\tt x} if
{\tt control\%second\_derivatives\_available} has been set to \true\.
It need not be set otherwise.

\end{description}

%%%%%%%%%%%%%%%%%%%%%% Reverse communication %%%%%%%%%%%%%%%%%%%%%%%%

\subsection{\label{reverse}Reverse Communication Information}

A positive value of {\tt inform\%status} on exit from
{\tt \packagename\_solve}
indicates that
\solver\ is seeking further information---this will happen
if the user has chosen not to evaluate function or
derivative values internally (see Section~\ref{fdv}).
The user should compute the required information and re-enter \solver\
with {\tt inform\%status} and all other arguments (except those specifically
mentioned below) unchanged.

Possible values of {\tt inform\%status} and the information required are
\begin{description}
\ittf{3.} The user should compute the objective function
     value $f(x)$ and its first derivative $f^{\prime}(x)$
     at the point $x$ indicated in {\tt x}.
     The required values should be set in {\tt f} and {\tt g}.
     respectively, and  {\tt data\%eval\_status} should be set to 0.
     If the user is unable to evaluate $f(x)$ or its
     first derivative---for instance, if the function is
     undefined at $x$---the user need not set {\tt f} or {\tt g},
     but should then set {\tt data\%eval\_status} to a non-zero value.
     This value can only occur when
     {\tt control\%second\_derivatives\_available = .FALSE.}

\ittf{4.} The user should compute the objective function
     value $f(x)$ and its first two derivatives $f^{\prime}(x)$ and
     $f^{\prime\prime}(x)$ at the point $x$ indicated in {\tt x}.
     The required values should be set in {\tt f}, {\tt g} and {\tt h},
     respectively, and  {\tt data\%eval\_status} should be set to 0.
     If the user is unable to evaluate $f(x)$ or either of its
     derivatives---for instance, if the function is
     undefined at $x$---the user need not set {\tt f}, {\tt g} or {\tt h},
     but should then set {\tt data\%eval\_status} to a non-zero value.
     This value can only occur when
     {\tt control\%second\_derivatives\_available = .TRUE.}
\end{description}
Other values are reserved for future developments.

%%%%%%%%%%%%%%%%%%%%%% Warning and error messages %%%%%%%%%%%%%%%%%%%%%%%%

\galerrors
A negative value of {\tt inform\%status} on exit from
{\tt \packagename\_solve}
or
{\tt \packagename\_terminate}
indicates that an error has occurred. No further calls should be made
until the error has been corrected. Possible values are:

\begin{description}

\itt{\galerrallocate.} A workspace allocation error occurred.
A message indicating the offending
array is written on unit {\tt control\%error}, and the returned allocation
status and a string containing the name of the offending array
are held in {\tt inform\%alloc\_\-status}
and {\tt inform\%bad\_alloc}, respectively.

\itt{\galerrdeallocate.} A workspace deallocation error occurred.
A message indicating the offending
array is written on unit {\tt control\%error} and the returned allocation
status and a string containing the name of the offending array
are held in {\tt inform\%alloc\_\-status}
and {\tt inform\%bad\_alloc}, respectively.

\itt{\galerrbadbounds.} The bound constraints are inconsistent.

\itt{\galerrunbounded.}  The objective function appears to be unbounded
 from below on the feasible set.

\itt{\galerrmaxiterations.} Too many iterations have been performed.
  This may happen if
    {\tt control\%maxit} is too small, but may also be symptomatic of
    a badly scaled problem.

\itt{\galerrcpulimit.} The elapsed CPU or system clock time limit has been
    reached. This may happen if either {\tt control\%cpu\_time\_limit} or
    {\tt control\%clock\_time\_limit} is too small, but may also be symptomatic
    of a badly scaled problem.

\itt{\galerralive.} The user has forced termination of \solver\
     by removing the file named {\tt control\%a\-live\_file} from
     unit {\tt control\%alive\_unit}.

\end{description}

%%%%%%%%%%%%%%%%%%%%%% Further features %%%%%%%%%%%%%%%%%%%%%%%%

\galfeatures
\noindent In this section, we describe an alternative means of setting
control parameters, that is components of the variable {\tt control} of type
{\tt \packagename\_control\_type}
(see Section~\ref{typecontrol}),
by reading an appropriate data specification file using the
subroutine {\tt \packagename\_read\_specfile}. This facility
is useful as it allows a user to change  {\tt \packagename} control parameters
without editing and recompiling programs that call {\tt \packagename}.

A specification file, or specfile, is a data file containing a number of
"specification commands". Each command occurs on a separate line,
and comprises a "keyword",
which is a string (in a close-to-natural language) used to identify a
control parameter, and
an (optional) "value", which defines the value to be assigned to the given
control parameter. All keywords and values are case insensitive,
keywords may be preceded by one or more blanks but
values must not contain blanks, and
each value must be separated from its keyword by at least one blank.
Values must not contain more than 30 characters, and
each line of the specfile is limited to 80 characters,
including the blanks separating keyword and value.

The portion of the specification file used by
{\tt \packagename\_read\_specfile}
must start
with a "{\tt BEGIN \packagename}" command and end with an
"{\tt END}" command.  The syntax of the specfile is thus defined as follows:
\begin{verbatim}
  ( .. lines ignored by UGO_read_specfile .. )
    BEGIN UGO
       keyword    value
       .......    .....
       keyword    value
    END
  ( .. lines ignored by UGO_read_specfile .. )
\end{verbatim}
where keyword and value are two strings separated by (at least) one blank.
The ``{\tt BEGIN \packagename}'' and ``{\tt END}'' delimiter command lines
may contain additional (trailing) strings so long as such strings are
separated by one or more blanks, so that lines such as
\begin{verbatim}
    BEGIN UGO SPECIFICATION
\end{verbatim}
and
\begin{verbatim}
    END UGO SPECIFICATION
\end{verbatim}
are acceptable. Furthermore,
between the
``{\tt BEGIN \packagename}'' and ``{\tt END}'' delimiters,
specification commands may occur in any order.  Blank lines and
lines whose first non-blank character is {\tt !} or {\tt *} are ignored.
The content
of a line after a {\tt !} or {\tt *} character is also
ignored (as is the {\tt !} or {\tt *}
character itself). This provides an easy manner to "comment out" some
specification commands, or to comment specific values
of certain control parameters.

The value of a control parameters may be of three different types, namely
integer, logical or real.
Integer and real values may be expressed in any relevant Fortran integer and
floating-point formats (respectively). Permitted values for logical
parameters are "{\tt ON}", "{\tt TRUE}", "{\tt .TRUE.}", "{\tt T}",
"{\tt YES}", "{\tt Y}", or "{\tt OFF}", "{\tt NO}",
"{\tt N}", "{\tt FALSE}", "{\tt .FALSE.}" and "{\tt F}".
Empty values are also allowed for
logical control parameters, and are interpreted as "{\tt TRUE}".

The specification file must be open for
input when {\tt \packagename\_read\_specfile}
is called, and the associated device number
passed to the routine in device (see below).
Note that the corresponding
file is {\tt REWIND}ed, which makes it possible to combine the specifications
for more than one program/routine.  For the same reason, the file is not
closed by {\tt \packagename\_read\_specfile}.

\subsubsection{To read control parameters from a specification file}
\label{readspec}

Control parameters may be read from a file as follows:
\hskip0.5in
\def\baselinestretch{0.8} {\tt
\begin{verbatim}
     CALL UGO_read_specfile( control, device )
\end{verbatim}
}
\def\baselinestretch{1.0}

\begin{description}
\itt{control} is a scalar \intentinout argument of type
{\tt \packagename\_control\_type}
(see Section~\ref{typecontrol}).
Default values should have already been set, perhaps by calling
{\tt \packagename\_initialize}.
On exit, individual components of {\tt control} may have been changed
according to the commands found in the specfile. Specfile commands and
the component (see Section~\ref{typecontrol}) of {\tt control}
that each affects are given in Table~\vref{specfile}.

\bctable{|l|l|l|}
\hline
  command & component of {\tt control} & value type \\
\hline
  {\tt error-printout-device} & {\tt \%error} & integer \\
  {\tt printout-device} & {\tt \%out} & integer \\
  {\tt print-level} & {\tt \%print\_level} & integer \\
  {\tt start-print} & {\tt \%start\_print} & integer \\
  {\tt stop-print} & {\tt \%stop\_print} & integer \\
  {\tt iterations-between-printing} & {\tt \%print\_gap} & integer \\
  {\tt maximum-number-of-iterations} & {\tt \%maxit} & integer \\

  {\tt number-of-initial-points} & {\tt  initial\_points} & integer \\
  {\tt block-of-storage-allocated} & {\tt storage\_increment} & integer \\
  {\tt out-of-core-buffer} & {\tt buffer} & integer \\
  {\tt lipschitz-estimate-used} & {\tt lipschitz\_estimate\_used} & integer \\
  {\tt next-interval-selection-method} & {\tt next\_interval\_selection} & integer \\
  {\tt refine-with-newton-iterations} & {\tt refine\_with\_newton} & integer \\
  {\tt alive-device} & {\tt \%alive\_unit} & integer \\
  {\tt stop\_length} & {\tt maximum-interval-length-required} & real \\
  {\tt small\_g\_for\_newton} & {\tt try-newton-tolerance} & real \\
  {\tt small\_g} & {\tt sufficient-gradient-tolerance} & real \\
  {\tt obj\_sufficient} & {\tt sufficient-objective-value} & real \\
  {\tt global-lipschitz-constant} & {\tt global\_lipschitz\_constant} & real \\
  {\tt reliability\_parameter} & {\tt lipschitz-reliability-parameter} & real \\
  {\tt lipschitz\_lower\_bound} & {\tt lipschitz-lower-bound} & real \\
  {\tt maximum-cpu-time-limit} & {\tt \%cpu\_time\_limit} & real \\
  {\tt maximum-clock-time-limit} & {\tt \%clock\_time\_limit} & real \\
  {\tt space-critical}   & {\tt \%space\_critical} & logical \\
  {\tt deallocate-error-fatal}   & {\tt \%deallocate\_error\_fatal} & logical \\
  {\tt alive-filename} & {\tt \%alive\_file} & character \\
\hline

\ectable{\label{specfile}Specfile commands and associated
components of {\tt control}.}

\itt{device} is a scalar \intentin argument of type default \integer,
that must be set to the unit number on which the specfile
has been opened. If {\tt device} is not open, {\tt control} will
not be altered and execution will continue, but an error message
will be printed on unit {\tt control\%error}.

\end{description}

%%%%%%%%%%%%%%%%%%%%%% Information printed %%%%%%%%%%%%%%%%%%%%%%%%

\galinfo
If {\tt control\%print\_level} is positive, information about the progress
of the algorithm will be printed on unit {\tt control\-\%out}.
If {\tt control\%print\_level} $= 1$, a single line of output will be produced
for each iteration of the process.
This will include the value of the current iterate $x$ and its corresponding
objective function  and gradient values, as well as the current best $x$ and
objective values.
If {\tt control\%print\_level} $=2$, a final summary of the intervals
considered is given, while if {\tt control\%print\_level} $>2$,
additional debugging information will be given.

%%%%%%%%%%%%%%%%%%%%%% GENERAL INFORMATION %%%%%%%%%%%%%%%%%%%%%%%%

\galgeneral

\galcommon None.
\galworkspace Provided automatically by the module.
\galroutines None.
\galmodules {\tt \packagename\_solve} calls the \galahad\ packages
{\tt GALAHAD\_CLOCK},
{\tt GALAHAD\_SY\-M\-BOLS},
{\tt GALAHAD\_SPECFILE},
{\tt GALAHAD\_SPACE},
{\tt GALAHAD\_STRINGS},
and
{\tt GALAHAD\_USERDATA}.
\galio Output is under control of the arguments
 {\tt control\%error}, {\tt control\%out} and {\tt control\%print\_level}.
\galrestrictions None.
\galportability ISO Fortran~95 + TR 15581 or Fortran~2003.
The package is thread-safe.

%%%%%%%%%%%%%%%%%%%%%% METHOD %%%%%%%%%%%%%%%%%%%%%%%%

\galmethod

The algorithm starts by splitting the interval $[x^l,x^u]$ into a specified
number of subintervals $[x_i,x_{i+1}]$ of equal length, and evaluating
$f$ and its derivatives at each $x_i$. A surrogate (approximating)
lower bound function is constructed on each subinterval using the
function and derivative values at each end, and an estimate of the
first- and second-derivative Lipschitz constant. This surrogate is
minimized, the true objective evaluated at the best predicted point,
and the corresponding interval split again at this point.
Any interval whose surrogate lower bound value exceeds an evaluated
actual value is discarded. The method continues until only one interval
of a maximum permitted width remains.

\galreferences
\vspace*{1mm}

\noindent
 Many ingredients in the algorithm are based on the paper
\vspace*{1mm}

\noindent
Daniela Lera and Yaroslav D. Sergeyev,
"Acceleration of univariate global optimization algorithms working with
 Lipschitz functions and Lipschitz first derivatives"
SIAM J. Optimization Vol. 23, No. 1, pp. 508–529 (2013)

\noindent
but adapted to use second derivatives.

%%%%%%%%%%%%%%%%%%%%%% EXAMPLE %%%%%%%%%%%%%%%%%%%%%%%%

\galexamples
Suppose we wish to minimize the parametric objective function
$f(\bmx) =  x^2 \sin(ax)$
when the parameter $a$ takes the value 10, and $x$ is required to satisfy
the  bounds $-1 \leq x \leq 2$.
We may use the following code:

{\tt \small
\VerbatimInput{\packageexample}
}
\noindent
Notice how the parameter $a$ is passed to the function evaluation
routines via the {\tt real} component of the derived type {\tt userdata}.
The code produces the following output:
{\tt \small
\VerbatimInput{\packageresults}
}
\noindent
The same problem may be solved by returning to the user to provide
the desired function and derivative values as follows:

{\tt \small
\VerbatimInput{\packageexampleb}
}
\noindent
This produces the same output.
%following output:
%{\tt \small
%\VerbatimInput{\packageresultsb}
%}
%\noindent

\end{document}
