\documentclass{galahad}

% set the package name

\newcommand{\package}{scu}
\newcommand{\packagename}{SCU}
\newcommand{\fullpackagename}{\libraryname\-\_\-\packagename}

\begin{document}

\galheader

%%%%%%%%%%%%%%%%%%%%%% SUMMARY %%%%%%%%%%%%%%%%%%%%%%%%

\galsummary

{\tt \fullpackagename} is a suite of Fortran procedures for 
computing the 
the {\bf solution to an extended system of $n + m$ 
sparse real linear equations in $n  +  m$ unknowns,} 
\eqn{1}{\mat{cc}{ \bmA & \bmB \\ \bmC & \bmD } \vect{\bmx_1 \\ \bmx_2}
 =  \vect{\bmb_1 \\ \bmb_2}.}
in the case where the $n$ by $n$ matrix $\bmA$ is nonsingular 
and solutions to the systems 
\disp{\bmA \bmx  =  \bmb \tim{and} \bmA^T \bmy  =  \bmc}
may be obtained from an external source, such as an existing 
factorization.  The subroutine uses reverse communication to obtain 
the solution to such smaller systems.  The method makes use of 
the Schur complement matrix 
\disp{\bmS  =  \bmD  -  \bmC    \bmA^{-1} \bmB.}
The Schur complement is stored and factorized as a dense matrix 
and the subroutine is thus appropriate only if there is 
sufficient storage for this matrix. Special advantage is taken 
of symmetry and definiteness in the coefficient matrices. 
Provision is made for introducing additional rows and columns 
to, and removing existing rows and columns from, the extended 
matrix. 
 
%%%%%%%%%%%%%%%%%%%%%% attributes %%%%%%%%%%%%%%%%%%%%%%%%

\galattributes
\galversions{\tt  \fullpackagename\_single, \fullpackagename\_double},
\galuses {\tt \_ROT}, {\tt \_ROTG}. 
\galdate October 2001.
\galorigin N. I. M. Gould, Rutherford Appleton Laboratory, and
Ph. L. Toint, University of Namur, Belgium.
\gallanguage Fortran~95 + TR 15581 or Fortran~2003. 

%%%%%%%%%%%%%%%%%%%%%% HOW TO USE %%%%%%%%%%%%%%%%%%%%%%%%

\galhowto

Access to the package requires a {\tt USE} statement such as

\medskip\noindent{\em Single precision version}

\hskip0.5in {\tt USE \fullpackagename\_single}

\medskip\noindent{\em Double precision version}

\hskip0.5in {\tt USE  \fullpackagename\_double}

\medskip

\noindent
If it is required to use both modules at the same time, the derived types 
{\tt \packagename\_matrix\_type}, 
{\tt \packagename\_inform\_type}, and
{\tt \packagename\_data\_type} 
(Section~\ref{galtypes}),
and the subroutines
{\tt \packagename\_factorize},  
{\tt \packagename\_solve},  
{\tt \packagename\_append},  
{\tt \packagename\_delete}, 
and 
{\tt \packagename\_terminate} 
(Section~\ref{galarguments})
must be renamed on one of the {\tt USE} statements.

%%%%%%%%%%%%%%%%%%%%%% derived types %%%%%%%%%%%%%%%%%%%%%%%%

\galtypes
Three derived data types are accessible from the package.

%%%%%%%%%%% matrix type %%%%%%%%%%%

\subsubsection{The derived data type for holding matrix information}
\label{typematrix}
The derived data type {\tt \packagename\_matrix\_type} is used to hold 
data about the extended matrix and its factors. The components of 
{\tt \packagename\_matrix\_type} 
are:

\begin{description}

\ittf{n} is a scalar of type default \integer, that holds the value of 
$n$, the dimension of the matrix $ \bmA$. 
 
\ittf{m} is a scalar of type default \integer, that holds the value of  
$m$, dimension of the matrix $\bmD$. 
 
\itt{m\_max} is an scalar of type default \integer,
that holds the dimension of the largest possible matrix $\bmD$ to be permitted
by the package. 

\itt{class} is a scalar of type default \integer,
that indicates the type of matrix to be processed. Permitted values are: 

\begin{description} 
\itt{1} the extended matrix is unsymmetric, 
\itt{2} the extended matrix is symmetric, 
\itt{3} the extended matrix is symmetric and the Schur complement matrix 
$\bmS$ is known to be positive definite. 
\itt{4} the extended matrix is symmetric and the Schur complement matrix 
$\bmS$ is known to be negative definite. 
\end{description} 

\itt{BD\_val} is a rank-one allocatable of  
type default \realdp, 
that holds the values of the entries in the partitioned matrix 
\eqn{2}{\vect{\bmB \\ \bmD_U},}
where $\bmD_U$ is the upper triangular part of $\bmD$, ie, 
$( \bmD_U )_{ij}  =  ( \bmD )_{ij} $ if $i  \leq  j$ and zero otherwise. 
The entries must be ordered by columns, with the entries in each column 
contiguous and those of column 
$j$ preceding those of column $j+1$ $(j = 1, ...., m)$. The ordering 
within each column is unimportant. 
 
\itt{BD\_row} is a rank-one allocatable of type default \integer, that holds 
the row indices of the corresponding entries in {\tt BD\_val.} 
 
\itt{BD\_col\_start} is a rank-one allocatable of type default \integer, that  
must be set so that {\tt BD\_col\_start(j)} holds the positions 
in the arrays {\tt BD\_val} and {\tt BD\_row} of the first entry in column $j$ 
$(j = 1, ...., m)$. {\tt BD\_col\_start(m+1)} must be set to 
the number of entries in the matrix (2) plus one. 
 
\itt{CD\_val} is a rank-one allocatable of type default  
\realdp, that need not be 
set in the symmetric case. In the unsymmetric case, it 
holds the values of the entries in the partitioned matrix 
\eqn{3}{ ( \bmC \;\;   \bmD_L ),}
where $\bmD_L$ is the strict lower triangular part of $\bmD$, ie, 
$( \bmD_L )_{ij}  =  ( \bmD )_{ij} $ if $i  >  j$ and zero 
otherwise. 
The entries must be ordered by rows, with the entries in each row 
contiguous and those of row 
$i$ preceding those of row $i+1$ $(i = 1, ...., m)$. 
The ordering within each row is unimportant. 
 
\itt{CD\_col} is a rank-one allocatable of type default \integer, 
that need not be set in the symmetric case. In the unsymmetric case, it holds 
the column indices of the corresponding entries in {\tt CD\_val}. 
 
\itt{CD\_row\_start} is a rank-one allocatable of type default \integer, 
that need not be set in the symmetric case. In the unsymmetric case, it  
must be set so that {\tt CD\_row\_start(i)} points to the  
positions in the arrays {\tt CD\_val} and {\tt CD\_col} of the first entry in row $i$ 
$(i = 1, ...., m)$. {\tt CD\_row\_start(m+1)} must be set to 
the number of entries in the matrix \req{3} plus one. 
 
\end{description}

%%%%%%%%%%% data type %%%%%%%%%%%

\subsubsection{The derived data type for holding problem data}\label{typedata}
The derived data type 
{\tt \packagename\_data\_type} 
is used to hold all the data for a particular problem between calls of 
{\tt \packagename} procedures. 
This data should be preserved, untouched, from the initial call to 
{\tt \packagename\_factorize}
to the final call to
{\tt \packagename\_terminate}.
All components of {\tt \packagename\_data\_type} are private.

%%%%%%%%%%% info type %%%%%%%%%%%

\subsubsection{The derived data type for holding informational
 parameters}\label{typeinfo}
The derived data type 
{\tt \packagename\_inform\_type} 
is used to hold parameters that give information about the factorization.
The components of
{\tt \packagename\_inform\_type} 
are:

\begin{description}
\itt{alloc\_status} is a scalar of type default \integer, that contains 
the return status from the last attempted internal  
workspace array allocation or deallocation. 
A non-zero value indicates that the allocation or deallocation was
 unsuccessful, and corresponds to the {\tt STAT=} value on the user's system. 
Consult local compiler documentation for further details. 
 
\itt{inertia} is a rank-one array of length 3 and type default \integer, 
that holds the inertia of $\bmS$ when the extended matrix is symmetric. 
Specifically, {\tt inertia(i),} {\tt i=1,2,3},  
give the number of positive, negative and zero eigenvalues of  
$\bmS$ respectively. 
 
\end{description}

\noindent
For backward compatibility with an earlier version of the package, there
is an equivalent type {\tt \packagename\_info\_type}, and the two names
may be used interchangeably.

%%%%%%%%%%%%%%%%%%%%%% argument lists %%%%%%%%%%%%%%%%%%%%%%%%

\galarguments
To solve the extended system, 
%the user must first call 
%{\tt \packagename\_initialize} to set up the internal data structures that 
%will be used 
%to hold the factors of the Schur complement matrix $\bmS$. 
%Next, 
the user must first call {\tt \packagename\_factorize} to form and factorize 
the Schur complement, 
and then call {\tt \packagename\_solve} to compute the solution to the extended 
system. The solution of additional extended systems, with the 
same coefficient matrix but different right-hand sides, may 
be found by further calls to 
{\tt \packagename\_solve.} 

The solution of further-extended systems of the form 
\disp{\mat{ccc}{ \bmA & \bmB & \bmc_1 \\ \bmC & \bmD & \bmc_2 \\ 
 \bmr_1^T &  \bmr_2^T & d } \vect{\bmx_1 \\ \bmx_2 \\ x_{n+m+1}} = 
 \vect{\bmb_1 \\ \bmb_2 \\ b_{n+m+1}}.}
may be found by firstly calling {\tt \packagename\_append} to update the 
existing factorization of $\bmS$ 
(obtained from {\tt \packagename\_factorize)} to give that of 
the Schur complement of $ \bmA$ 
in the further-extended coefficient matrix and then by calling 
{\tt \packagename\_solve.} 
Likewise, the solution of extended systems of the form 
\disp{\mat{cc}{ \bmA & \bar{\bmB} \\ \bar{\bmC} & \bar{\bmD} }
 \vect{\bmx_1 \\ \bmx_2} = \vect{\bmb_1 \\ \bmb_2}.}
in which a row and column are removed from the 
coefficient matrix of \req{1}, 
may be found by firstly calling {\tt \packagename\_delete} to update the 
existing factorization of $\bmS$ 
(obtained from {\tt \packagename\_factorize)} to give that of 
the Schur complement of $ \bmA$ 
in the new extended coefficient matrix and then once again by calling 
{\tt \packagename\_solve.} 

Finally, the user may call {\tt \packagename\_terminate} 
to deallocate any workspace used to hold the factors of $\bmS$. 

We use square brackets {\tt [ ]} to indicate \optional arguments.

%%%%%% initialization subroutine %%%%%%

%\subsubsection{The initialization stage}\label{subinit}

%Internal data structures required to hold the factors of 
%the (sequence of) Schur complement(s) must be initialized
%as follows: 
%\vspace*{1mm}%

%\hskip0.5in 
%{\tt CALL \packagename\_initialize( data )}

%\begin{description}

%\itt{data} is a scalar \intentout\ argument of type 
%{\tt \packagename\_data\_type} 
%(see Section~\ref{typedata}) used to hold the factors of $\bmS$. 
 
%\end{description}

%%%%%%%%% factorization subroutine %%%%%%

\subsubsection{The factorization stage}
The Schur complement matrix may be factorized as follows:
\vspace*{1mm}

\hskip0.5in 
{\tt CALL \packagename\_factorize( matrix, data, VECTOR, status, info )}

\vspace*{-3mm}
\begin{description}

\itt{matrix} is a scalar \intentinout\ argument of type 
{\tt \packagename\_matrix\_type} 
(see Section~\ref{typematrix}). 
The following components are used by {\tt \packagename\_factorize} : 
 
\begin{description} 
\itt{n} must be set to $n$, the dimension of the matrix $ \bmA$. 
It is unchanged by the subroutine. {\bf Restriction:} ${\tt n} \geq 0$.
 
\itt{m\_max} must be set to the 
dimension of the largest matrix $\bmD$ that will be allowed 
by the package. Internal workspace of dimension 
{\tt m\_max} $\ast$ {\tt (m\_max+3)/2} 
(or {\tt 3m\_max} $\ast$ {\tt (m\_max+1)/2} if {\tt class = 1}) 
will be used by the package. Thus care should be taken not to set {\tt m\_max} 
larger than is absolutely necessary to account for the sequence of  
extended matrices likely to be encountered. 
{\bf Restriction:} ${\tt m\_max} \geq 0$.
 
\itt{m} must be set to $m$, dimension of the matrix $\bmD$. 
It is unchanged by the subroutine. 
{\bf Restriction:} ${\tt 0} \leq {\tt m} \leq {\tt m\_max}$.
 
\itt{class} must be set to indicate the type of matrix that will be
processed (see Section~\ref{typematrix}). 
If $\bmS$ is known 
to be (positive or negative) definite throughout the sequence 
of extended matrices to be considered, it is more efficient to set 
{\tt class} $= 3$ or 4. Likewise, if the matrix is known 
to be symmetric throughout the sequence 
of extended matrices to be considered, it is more efficient to set 
{\tt class} $> 1$.  
{\bf Restriction:} $1 \leq  {\tt class} \leq 4$. 
 
\itt{BD\_val} and {\tt BD\_row} must be set as described in 
Section~\ref{typematrix}  to hold the values and row indices, respectively, 
of the matrix \req{2}.  Before use, the arrays must be {\tt ALLOCATE}d 
to be of sufficient length to hold any matrix of the form \req{2} 
to be encountered in the sequence of 
extended matrices to be considered. Any elements in {\tt BD\_val} and 
{\tt BD\_row} 
that lie below the diagonal of $\bmD$ will be removed by 
{\tt \packagename\_factorize}. 
 
\itt{BD\_col\_start} must be set as described in Section~\ref{typematrix}
to hold the 
positions of the start of the columns of \req{2}, as well as to the first 
position past the end of the last column. 
Before use, this array must be {\tt ALLOCATE}d to be of length at least 
{\tt m\_max+1}. 
The values of this array may be altered if entries that lie 
below the diagonal of $\bmD$ are removed by {\tt \packagename\_factorize}. 
 
\itt{CD\_val} and {\tt CD\_col}.   
When {\tt class=1}, 
these must be set as described in Section~\ref{typematrix} to hold the 
values and column indices, respectively, of the matrix \req{3}.  
Before use, the arrays must be {\tt ALLOCATE}d 
to be of sufficient length to hold any matrix of the form \req{3} 
to be encountered in the sequence of 
extended matrices to be considered. Any elements in 
{\tt CD\_val} and {\tt CD\_col} 
that lie on or to the right of the diagonal of $\bmD$ will be removed by  
{\tt \packagename\_factorize}. 
This component need not be {\tt ALLOCATE}d or set if {\tt class} $> 1$. 
 
\itt{CD\_row\_start}. When {\tt class=1}, 
this must be set as described in Section~\ref{typematrix} to hold the 
positions of the start of the rows in \req{3}, as well as to the first 
position past the end of the last row. 
Before use, this array must be {\tt ALLOCATE}d to be of length at least 
{\tt m\_max+1}. 
The values of this array may be altered if entries that lie on or to the 
right of the diagonal of $\bmD$ are removed by {\tt \packagename\_factorize}. 
This component need not be {\tt ALLOCATE}d or set if {\tt class} $> 1$. 
 
\end{description} 
 
\itt{data} is a scalar \intentinout\ argument of type 
{\tt \packagename\_data\_type}. 
It must not have been altered since the last call to  
%{\tt \packagename\_initialize}, 
{\tt \packagename\_factorize}, 
{\tt \packagename\_append} or 
{\tt \packagename\_delete}. 
 
\itt{VECTOR} is an \intentinout\ rank-one array argument of length 
{\tt matrix\%n} 
and type \realdp, 
that needs not be set by 
the user on initial ({\tt status=1)} entry. If {\tt status} is greater than 
1 on exit, a re-entry must be made with {\tt VECTOR} set appropriately 
(see Section~\ref{galreverse}). 
 
\itt{status} is an scalar \intentinout\ argument of type 
default {\tt INTEGER,} that  
must be set by the user on initial input to 1. On output, the value of 
{\tt status} 
is used to request additional information, to signal an error 
in the input data or to indicate a successful call to the subroutine. 
A successful call is indicated by the exit value {\tt status=0}. 
For other values, see Sections~\ref{galreverse} and \ref{galerrors}. 
 
\itt{info} is a scalar \intentinout\ argument of type 
{\tt \packagename\_inform\_type} 
(see Section~\ref{typeinfo}). 
 
\end{description}

%%%%%%% solution subroutine %%%%%%

\subsubsection{The solution stage}
Solve the extended system of equations using 
the  factorization produced by a previous call to  
{\tt \packagename\_factorize}, {\tt \packagename\_append} 
or {\tt \packagename\_delete}, 
as follows: 
\vspace*{1mm}

\hskip0.5in 
{\tt CALL \packagename\_solve( matrix, data, RHS, X, VECTOR, status )}

\vspace*{-2mm}
\begin{description}
\itt{matrix} is a scalar \intentin\ argument of type 
{\tt \packagename\_matrix\_type} 
(see Section~\ref{typematrix}), that must not have been changed since the 
last call to 
{\tt \packagename\_factorize}, {\tt \packagename\_append} or 
{\tt \packagename\_delete}. 
 
\itt{data} is a scalar \intentinout\ argument of type 
{\tt \packagename\_data\_type}. 
It must not have been altered since the last call to  
%{\tt \packagename\_initialize}, 
{\tt \packagename\_factorize}, 
{\tt \packagename\_append} or 
{\tt \packagename\_delete}. 
 
\itt{RHS} is an \intentin\ rank-one array argument of length  
{\tt matrix\%n+matrix\%m} 
and type \realdp, 
that must be set on entry to the values of the right-hand-side vector 
\disp{\vect{ \bmb_1 \\ \bmb_2}}
of the extended 
system of equations. {\tt RHS} is not altered by the subroutine. 
 
\itt{X} is an \intentout\ rank-one array argument of length  
{\tt matrix\%n+matrix\%m} 
and type \realdp. 
On final ({\tt status=0)} exit, {\tt X} contains the values of the solution 
\disp{\vect{ \bmx_1 \\ \bmx_2}}
to the extended system of equations. 
 
\itt{VECTOR} is an \intentinout\ rank-one array argument of length 
{\tt matrix\%n} 
and type \realdp, that needs not be set by 
the user on initial ({\tt status=1)} entry. If {\tt status} is greater than 
1 on exit, a re-entry must be made with {\tt VECTOR} set appropriately 
(see Section~\ref{galreverse}). 
 
\itt{status} is an scalar \intentinout\ argument of type default 
{\tt INTEGER,} that  
must be set by the user on initial input to 1. On output, the value of 
{\tt status} 
is used to request additional information, to signal an error 
in the input data or to indicate a successful call to the subroutine. 
A successful call is indicated by the exit value {\tt status=0}. 
For other values, see Section~\ref{galreverse}. 

\end{description}

%%%%%%% updating subroutine %%%%%%

\subsubsection{The updating stage}
Call {\tt \packagename\_append} to extend the factorization of the Schur 
complement when a new row and column are appended to the 
extended matrix. Subsequent systems of equations with the larger 
coefficient matrix may then be solved by calls to {\tt \packagename\_solve.} 
Note in particular that the arrays {\tt matrix\_CD\_val}, 
{\tt matrix\_CD\_col}, {\tt matrix\_CD\_row\_start}, {\tt matrix\_BD\_val},  
{\tt matrix\_BD\_row} and {\tt matrix\_BD\_col\_start} 
must be sufficiently large to allow for the incoming row and column. 
The factorization may be extended as follows:
\vspace*{1mm}

\hskip0.5in 
{\tt CALL \packagename\_append( matrix, data, VECTOR, status, info )}

\vspace*{-2mm}
\begin{description}
\itt{matrix} is a scalar \intentinout\ argument of type 
{\tt \packagename\_matrix\_type} 
(see Section~\ref{typematrix}). The following components are used by 
{\tt \packagename\_append} : 
 
\begin{description} 
\ittf{n}, {\tt m\_max}, {\tt m} and {\tt class}.
These should not have been changed since the last call to 
{\tt \packagename\_factorize}, {\tt \packagename\_append} or 
{\tt \packagename\_delete}. 
On a successful exit ({\tt status} = 0) from {\tt \packagename\_append},  
{\tt n} will be unaltered and {\tt m} will have 
been increased by 1 to account for the appended row and column. 
{\bf Restriction:} ${\tt 0} \leq {\tt m} \leq {\tt m\_max}-1$.
 
\itt{BD\_val} and {\tt BD\_row}. These must be set as described  
in Section~\ref{typematrix} to hold the values and row indices, respectively, 
of the further extended matrix  
\eqn{4}{\mat{cc}{ \bmB & \bmc_1 \\ \bmD_U &  \bmc_2 \\ & d}.}
Specifically, the first {\tt col\_start(m+1)-1} elements 
of {\tt BD\_val} and {\tt BD\_row} should not have been changed since the 
last call to 
{\tt \packagename\_factorize}, {\tt \packagename\_append} or 
{\tt \packagename\_delete}, while 
elements {\tt col\_start(m+1)}, ... should contain 
the values and row indices of the appended column 
\disp{\vect{ \bmc_1 \\ \bmc_2 \\ d},}
the ordering within this column being unimportant. 
 
\itt{BD\_col\_start} must be set as described in 
Section~\ref{typematrix} to hold the 
positions of the start of the columns of \req{4}, as well as to the first 
position past the end of the last column. 
Specifically, the first {\tt m+1} elements 
of {\tt BD\_col\_start} should not have been changed since the last call to 
{\tt \packagename\_factorize}, {\tt \packagename\_append} or 
{\tt \packagename\_delete}, while 
{\tt BD\_col\_start(m+2)} should be set to the number of entries in 
\req{4} plus one. 
 
\itt{CD\_val} and {\tt CD\_col}. 
When {\tt \packagename\_factorize} was last called with {\tt matrix\%class=1}, 
these must be set as described in Section~\ref{typematrix} to hold the 
values and column indices respectively of the further extended matrix  
\eqn{5}{\mat{cc}{\bmC & \bmD_L \\ \bmr_1^T & \bmr_2^T}.}
Specifically, the first {\tt row\_start(m+1)-1} elements 
of {\tt CD\_val} and {\tt CD\_col} should not have been changed since the 
last call to 
{\tt \packagename\_factorize}, {\tt \packagename\_append} or 
{\tt \packagename\_delete}, while 
elements {\tt row\_start(m+1)}, ... should contain 
the values and column indices of the appended row 
\disp{( \bmr_1 \;\; \bmr_2^T ),}
the ordering within this row being unimportant. 
This component need not be {\tt ALLOCATE}d or set if
{\tt \packagename\_fac\-torize} was last called with {\tt matrix\%class} $> 1$. 
 
\itt{CD\_row\_start.} 
When {\tt \packagename\_factorize} was last called with {\tt matrix\%class=1}, 
this must be set as described in Section~\ref{typematrix} to hold the 
positions of the start of the rows of (6), as well as to the first 
position past the end of the last row. 
Specifically, the first {\tt m+1} elements 
of {\tt CD\_row\_start} should not have been changed since the last call to 
{\tt \packagename\_factorize}, {\tt \packagename\_append} or 
{\tt \packagename\_delete}, while 
{\tt CD\_row\_start(m+2)} should be set to the number of entries in 
\req{5} plus one. 
This component need not be {\tt ALLOCATE}d or set if 
{\tt \packagename\_factorize} was last called with {\tt matrix\%class} $> 1$. 
 
\end{description} 
 
\itt{data} is a scalar \intentinout\ argument of type 
{\tt \packagename\_data\_type}. 
It must not have been altered since the last call to  
%{\tt \packagename\_initialize}, 
{\tt \packagename\_factorize}, 
{\tt \packagename\_append} or 
{\tt \packagename\_delete}. 
 
\itt{VECTOR} is an \intentinout\ rank-one array argument of length 
{\tt matrix\%n} 
and type \realdp, that needs not be set by 
the user on initial ({\tt status=1)} entry. If {\tt status} is greater than 
1 on exit, a re-entry must be made with {\tt VECTOR} set appropriately 
(see Section~\ref{galreverse}). 
 
\itt{status} is an scalar \intentinout\ argument of type default 
{\tt INTEGER,} that  
must be set by the user on initial input to 1. On output, the value of 
{\tt status} 
is used to request additional information, to signal an error 
in the input data or to indicate a successful call to the subroutine. 
A successful call is indicated by the exit value {\tt status=0}. 
For other values, see Section~\ref{galreverse} and \ref{galerrors}. 
 
\itt{info} is a scalar \intentinout\ argument of type 
{\tt \packagename\_inform\_type} (see Section~\ref{typeinfo}). 
 
\end{description}

%%%%%%% deletion subroutine %%%%%%

\subsubsection{The deletion stage}
Call {\tt \packagename\_delete} to extend the factorization of the Schur 
complement when a row and column of the existing 
extended matrix are to be removed. 
Subsequent systems of equations with the smaller 
coefficient matrix may then be solved by calls to {\tt \packagename\_solve.} 
The data structures for holding $\bmB$, $\bmC$ and $\bmD$ will be 
automatically updated to account for the row and column deletions. 
Compute the factorization of the extended matrix 
following a row and column removal as follows:
\vspace*{1mm}

\hskip0.5in 
{\tt CALL \packagename\_delete( matrix, data, VECTOR, status, info,  
col\_del [, row\_del ] )}

\vspace*{-2mm}
\begin{description}
\itt{matrix} is a scalar \intentinout\ argument of type 
{\tt \packagename\_matrix\_type} 
(see Section~\ref{typematrix}). The following components are used by 
{\tt \packagename\_delete} : 
 
\begin{description} 
\ittf{n}, {\tt m\_max}, {\tt m}, {\tt class},
{\tt BD\_val}, {\tt BD\_row}, {\tt BD\_col\_start}, 
{\tt CD\_val}, {\tt CD\_col} and {\tt CD\_row\_start}.  
These should not have been changed since the last call to 
{\tt \packagename\_factorize}, {\tt \packagename\_append} or 
{\tt \packagename\_delete}. 
On a successful exit ({\tt status} = 0) from {\tt \packagename\_delete},  
{\tt n} will be unchanged, {\tt m} will have 
been decreased by 1 to account for the deleted row and column, and 
the contents of the arrays may have been changed. 
\end{description} 
 
\itt{data} is a scalar \intentinout\ argument of type 
{\tt \packagename\_data\_type}. 
It must not have been altered since the last call to  
%{\tt \packagename\_initialize}, 
{\tt \packagename\_factorize}, 
{\tt \packagename\_append} or 
{\tt \packagename\_delete}. 
 
\itt{VECTOR} is an \intentout\ rank-one array argument of length 
{\tt matrix\%n} 
and type \realdp, that is used as workspace. 
 
\itt{status} is an scalar \intentout\ argument of type default 
\integer. 
On output, the value of {\tt status} is used to signal an error 
in the input data or to indicate a successful call to the subroutine. 
A successful call is indicated by the exit value {\tt status=0}. 
For other values, see Section~\ref{galerrors}. 
 
\itt{info} is a scalar \intentinout\ argument of type 
{\tt \packagename\_inform\_type} 
(see Section~\ref{typeinfo}). 
 
\itt{col\_del} is an scalar \intentin\ argument of type 
default \integer. 
On input, this must be set by the user to the index of the column of 
\req{2} that is to be removed.  
{\bf Restriction:} ${\tt 1} \leq {\tt col\_del} \leq {\tt m}$.
 
\itt{row\_del} is an optional scalar \intentin\ argument of type  
default \integer. 
If {\tt \packagename\_factorize} was last called with {\tt matrix\%class=1}, 
this can be set by the user to the index of  
the row of \req{3} that is to be removed. If the argument is absent, or if 
{\tt  matrix\%class} $> 1$, row {\tt col\_del} will be removed. 
{\bf Restriction:} ${\tt 1} \leq {\tt row\_del} \leq {\tt m}$. 

\end{description}

%%%%%%% termination subroutine %%%%%%

\subsubsection{The final stage}
Deallocate the space required to hold the factors of the 
(sequence of) Schur complement(s) as follows:
\vspace*{1mm}

\hskip0.5in 
{\tt CALL \packagename\_terminate( data, status, info )}

\vspace*{-2mm}
\begin{description}
\itt{data} is a scalar \intentinout\ argument of type 
{\tt \packagename\_data\_type.} 
It must not have been altered since the last call to  
%{\tt \packagename\_initialize}, 
{\tt \packagename\_factorize}, 
{\tt \packagename\_append} or {\tt \packagename\_delete}. 
 
\itt{status} is an scalar \intentout\ argument of type default 
{\tt INTEGER,} that  
contains the exit status following a call to {\tt \packagename\_terminate.} 
A successful call is indicated by the exit value {\tt status=0}. 
For other values, see Section~\ref{galerrors}. 
 
\itt{info} is a scalar \intentout\ argument of type 
{\tt \packagename\_inform\_type} 
(see Section~\ref{typeinfo}). 
 

\end{description}

%%%%%%%%%%%%%%%%%%%%%% Re-entries %%%%%%%%%%%%%%%%%%%%%%%%

\galreverse
If one of the subroutines 
{\tt \packagename\_factorize},  
{\tt \packagename\_solve} or  
{\tt \packagename\_append} 
returns with a strictly possible value of {\tt status},  
the subroutine in question requires the user to manipluate the vector  
{\tt VECTOR}, and to re-enter with {\tt status} and all other arguments 
unchanged. 
Possible values of {\tt status} and their consequences are as follows: 

\begin{description} 
\itt{2} The user must obtain the solution to the system of equations 
$ \bmA \bmx  =  \bmb$. 
The particular vector $\bmb$ is returned in the array {\tt VECTOR;} 
the user must calculate $\bmx$ and pass this vector back in {\tt VECTOR.} 
\itt{3} The user must obtain the solution to the system of equations 
$ \bmA^T \bmy  =  \bmc$. 
The particular vector $\bmc$ is returned in the array {\tt VECTOR;} 
the user must calculate $\bmy$ and pass this vector back in {\tt VECTOR.} 
\end{description} 

%%%%%%%%%%%%%%%%%%%%%% Warning and error messages %%%%%%%%%%%%%%%%%%%%%%%%

\galerrors
A negative value of {\tt status} on exit from any of the preceding subroutines
indicates that an error has occurred. No further calls should be made
until the error has been corrected. Possible values are:

\begin{description}

\item{-1} One or more of the stated restrictions on the components 
$1 \leq  {\tt  matrix\%class} \leq 4$, 
${\tt matrix\%n} \geq 0$,
$0 \leq {\tt matrix\%m} \\ \leq {\tt matrix\%m\_max}$,
($0 \leq {\tt matrix\%m} \leq {\tt matrix\%m\_max}-1$ in 
{\tt \packagename\_append} ) 
$1 \leq {\tt col\_del} \leq {\tt matrix\%m}$ and 
$1 \leq {\tt row\_del} \leq {\tt matrix\%m}$
has been violated.  

\item{-2} The subroutine has been called with an initial value {\tt status} 
$\leq 0$. 

\item{-3} The factors of $\bmS$ have not yet been formed in {\tt data}. 
This indicates that either  
{\tt \packagename\_factorize} has not yet been called, or that the last call to 
{\tt \packagename\_factorize}, {\tt \packagename\_append} or 
{\tt \packagename\_delete} 
ended in a failure.  

\item{-4} One or more of the arrays {\tt matrix\%BD\_val}, 
{\tt matrix\%BD\_row} and 
{\tt matrix\%BD\_col\_start} has not been allocated. 

\item{-5} When the extended matrix is unsymmetric, one or more of the arrays  
{\tt matrix\%CD\_val}, {\tt matrix\%CD\_col} and 
{\tt matrix\%CD\_row\_start} has not been allocated. 

\item{-6} One or more of the arrays {\tt matrix\%BD\_val}, 
{\tt matrix\%BD\_row} and 
{\tt matrix\%BD\_col\_start} is not large enough. Check that the dimension 
of {\tt matrix\%BD\_col\_start} is no smaller than  
{\tt matrix\%m+1} 
({\tt matrix\%m+2} for {\tt \packagename\_a\-pp\-end} ), 
and that those of  
{\tt matrix\%BD\_val} and {\tt matrix\%BD\_row} are no smaller than 
{\tt matrix\%BD\_col\_start(mat\-rix\%m+1)-1}, and re-enter. 
({\tt matrix\%BD\_col\_start(matrix\%m+2)-1} for {\tt \packagename\_append} 
and {\tt matrix\%BD\_col\_st\-art(matrix\%m+1)+} 
{\tt |matrix\%col\_del-matrix\%row\_del|-1} for {\tt \packagename\_delete} ). 

\item{-7} When the extended matrix is unsymmetric, one or more of the arrays  
{\tt matrix\%CD\_val}, {\tt matrix\%CD\_col} and 
{\tt matrix\%CD\_row\_start} is not large enough. Check that the dimension 
of {\tt matrix\%CD\_row\_start} is no smaller than {\tt matrix\%m+1} 
({\tt matrix\%m+2} for {\tt \packagename\_append} ), and that those of  
{\tt matrix\%CD\_val} and {\tt matrix\%CD\_col} are no smaller than 
{\tt matrix\%CD\_row\_start(matrix\%m+1)-1} 
({\tt matrix\%CD\_row\_start(matrix\%m+2)-1} for {\tt \packagename\_app\-end} 
and {\tt matrix\%CD\_row\_start(matrix\%m+1)+} 
{\tt |matrix\%col\_del-matrix\%row\_del|-1} for {\tt \packagename\_delete} ). 

\item{-8} The value recorded in {\tt matrix\_m} does not correspond to the 
dimension of $\bmD$.  

\item{-9} The Schur complement matrix is singular; this has been 
detected during the {\bf QR } factorization of $\bmS$. 

\item{-10} The Schur complement matrix is expected to be positive definite, 
but this has been found not to be the case  
during the Cholesky factorization of $\bmS$. 

\item{-11} The Schur complement matrix is expected to be negative definite, 
but this has been found not to be the case  
during the Cholesky factorization of $-  \bmS$. 

\item{-12} An internal array allocation or deallocation failed.  
See {\tt info\%alloc\_status} for further details. 
 
\end{description}

%%%%%%%%%%%%%%%%%%%%%% Multiple updates %%%%%%%%%%%%%%%%%%%%%%%%

\subsection{Multiple updates}
Once {\tt \packagename\_append} or {\tt \packagename\_delete} 
has been used to update the factorization of 
the Schur complement matrix, it is as if the enlarged or 
reduced matrix were 
that originally factorized by {\tt \packagename\_factorize.} 
Consequently {\em sequences} of row and column additions and removals 
may be performed so long as sufficient storage is 
available. 

%%%%%%%%%%%%%%%%%%%%%% GENERAL INFORMATION %%%%%%%%%%%%%%%%%%%%%%%%

\galgeneral

\galcommon None.
\galworkspace Provided automatically by the module.
\galroutines the basic linear algebra subprograms (BLAS) 
       {\tt SROT} and {\tt SROTG} ({\tt DROT} and {\tt DROTG} in  
        {\tt GALAHAD\_\packagename\_DOUBLE} ) are called. 
\galmodules None.
\galio None.
\galrestrictions ${\tt matrix\%n} \geq {\tt 0}$, 
       $ 0 \leq {\tt matrix\%m} \leq {\tt matrix\%m\_max}$ and 
       ${\tt 1} \leq {\tt matrix\%class} \leq 4$. Also 
       ${\tt 1} \leq {\tt row\_del} \leq {\tt matrix\%m}$ and 
       ${\tt 1} \leq {\tt col\_del} \leq {\tt matrix\%m}$
       for {\tt \packagename\_delete.} 
\galportability ISO Fortran~95 + TR 15581 or Fortran~2003. 
The package is thread-safe.

%%%%%%%%%%%%%%%%%%%%%% METHOD %%%%%%%%%%%%%%%%%%%%%%%%

\galmethod
The subroutine {\tt \packagename\_factorize} forms the Schur complement 
$\bmS  =  \bmD  -  \bmC    \bmA^{-1} \bmB$ of $ \bmA$ 
in the extended matrix by repeated reverse communication to 
obtain the columns of 
$ \bmA^{-1} \bmB$. The Schur complement or its negative 
is then factorized 
into its {\bf QR } or, if possible, Cholesky factors. 

The subroutine {\tt \packagename\_solve} solves the extended system using 
the following well-known scheme: 

\begin{description} 
\item[(i)] Compute the solution to $ \bmA \bmu  =  \bmb_1$; 
\item[(ii)] Compute $\bmx_2$ from $ \bmS \bmx_2  =  \bmb_2  -  \bmC \bmu$; 
\item[(iii)] Compute the solution to $ \bmA \bmv  =  \bmB \bmx_2$; and 
\item[(iv)] Compute $\bmx_1  =  \bmu  -  \bmv$. 
\end{description} 


The subroutines {\tt \packagename\_append} and {\tt \packagename\_delete} 
compute the factorization 
of the Schur complement after a row and column have been appended 
to, and removed from, 
the extended matrix, respectively. 
The existing factorization is updated 
to obtain the new one; this is normally more efficient than 
forming the factorization from scratch. 
 
%%%%%%%%%%%%%%%%%%%%%% EXAMPLE %%%%%%%%%%%%%%%%%%%%%%%%

\galexample
As a simple example, suppose we wish to solve the system of 
equations 
\disp{\mat{ccccccc}{ 1 & 0 & 0 & 0 & 0 & 1 & 0 \\
               0 & 2 & 0 & 0 & 0 & 1 & 0 \\
               0 & 0 & 3 & 0 & 0 & 1 & 0 \\
               0 & 0 & 0 & 4 & 0 & 1 & 0 \\
               0 & 0 & 0 & 0 & 5 & 1 & 1 \\
               1 & 1 & 1 & 1 & 1 & 1 & 2 \\
               1 & 0 & 1 & 0 & 1 & 3 & 4 }
\vect{x_1 \\ x_2 \\ x_3 \\ x_4 \\ x_5 \\ x_6 \\ x_7} = 
\vect{2  \\ 3  \\ 4  \\ 5  \\ 7  \\ 8  \\ 10 }.}
Notice that the leading 5 by 5 coefficient matrix is diagonal and 
hence easily invertible. So we might choose $n = 5$, $m = 2$, 
and use {\tt \packagename\_factorize}/{\tt \packagename\_solve} to find the required 
solution. As the matrix is unsymmetric, we must set {\tt matrix\%class=1}.

Now suppose that we have solved this system as described and are now 
confronted with the further system 
\disp{\mat{cccccccc}{ 1 & 0 & 0 & 0 & 0 & 1 & 0 & 1 \\
               0 & 2 & 0 & 0 & 0 & 1 & 0 & 0 \\
               0 & 0 & 3 & 0 & 0 & 1 & 0 & 0 \\
               0 & 0 & 0 & 4 & 0 & 1 & 0 & 0 \\
               0 & 0 & 0 & 0 & 5 & 1 & 1 & 0 \\
               1 & 1 & 1 & 1 & 1 & 1 & 2 & 1 \\
               1 & 0 & 1 & 0 & 1 & 3 & 4 & 0 \\
               1 & 0 & 0 & 0 & 0 & 0 & 1 & 1 }
\vect{x_1 \\ x_2 \\ x_3 \\ x_4 \\ x_5 \\ x_6 \\ x_7 \\ x_8 } = 
\vect{5  \\ 5  \\ 4  \\ 5  \\ 7  \\ 12  \\ 12  \\ 4}.}
Rather than applying {\tt \packagename\_factorize} /{\tt \packagename\_solve} 
with $n = 5$, $m = 3$, we note 
that the new coefficient matrix differs from the old one merely 
in having 
an extra row and column. Thus, we can use 
{\tt \packagename\_append} with $n = 5$, $m = 2$ to update the existing factorization 
and then call 
{\tt \packagename\_solve} with $n = 5$, $m = 3$ to calculate the desired solution. 

Finally, suppose that we have solved this second system and now wish 
to solve 
\disp{\mat{ccccccc}{ 1 & 0 & 0 & 0 & 0 & 1 & 1 \\
               0 & 2 & 0 & 0 & 0 & 1 & 0 \\
               0 & 0 & 3 & 0 & 0 & 1 & 0 \\
               0 & 0 & 0 & 4 & 0 & 1 & 0 \\
               0 & 0 & 0 & 0 & 5 & 1 & 0 \\
               1 & 0 & 1 & 0 & 1 & 3 & 0 \\
               1 & 0 & 0 & 0 & 0 & 0 & 1 }
\vect{x_1 \\ x_2 \\ x_3 \\ x_4 \\ x_5 \\ x_6 \\ x_7} = 
\vect{3  \\ 5  \\ 4  \\ 5 \\ 6  \\ 6  \\ 2}.}
Again, rather than applying {\tt \packagename\_factorize} /{\tt \packagename\_solve} 
with $n = 5$, $m = 2$, we note 
that this third coefficient matrix is the second one with its 
$n + 1$-st row and $n + 2$-nd column removed. Thus, we can use 
{\tt \packagename\_delete} with $n = 5$, $m = 3$ to update the existing factorization 
and then call 
{\tt \packagename\_solve} with $n = 5$, $m = 2$ to calculate the desired solution. 

To carry out these calculations, we might use the following piece of code. 
Notice how the extra row and column for the second problem are 
simply introduced at the end of the existing data structures. 
{\tt
\begin{verbatim}
   PROGRAM GALAHAD_SCU_EXAMPLE 
   USE GALAHAD_SCU_DOUBLE                    ! double precision version
   IMPLICIT NONE 
   INTEGER, PARAMETER :: wp = KIND( 1.0D+0 ) ! set precision
   INTEGER :: i, row_del, col_del, status 
   INTEGER, PARAMETER :: n = 5, m = 2, mmax = m + 1 
   TYPE ( SCU_matrix_type ) :: mat 
   TYPE ( SCU_data_type ) :: data 
   TYPE ( SCU_info_type ) :: info 
   REAL ( KIND = wp ) :: X1( n + m ), RHS1( n + m )
   REAL ( KIND = wp ) :: X2( n + m + 1 ), RHS2( n + m + 1 )
   REAL ( KIND = wp ) :: X3( n + m ), RHS3( n + m ), VECTOR( n ) 
   mat%m_max = mmax ; mat%class = 1 
   mat%n = n ; mat%m = m 
   ALLOCATE ( mat%BD_val( 15 ), mat%BD_row( 15 ), mat%BD_col_start( mmax+1 ),& 
              mat%CD_val( 13 ), mat%CD_col( 13 ), mat%CD_row_start( mmax+1 ) ) 
   mat%BD_col_start( : 3 ) = (/1, 7, 10/) 
   mat%CD_row_start( : 3 ) = (/1, 6, 10/) 
   mat%BD_row( : 9 ) = (/1, 2, 3, 4, 5, 6, 5, 6, 7/) 
   mat%BD_val( : 9 ) = (/1.0_wp, 1.0_wp, 1.0_wp, 1.0_wp, & 
                     1.0_wp, 1.0_wp, 1.0_wp, 2.0_wp, 4.0_wp/) 
   mat%CD_col( : 9 ) = (/1, 2, 3, 4, 5, 1, 3, 5, 6/) 
   mat%CD_val( : 9 ) = (/1.0_wp, 1.0_wp, 1.0_wp, 1.0_wp, & 
                       1.0_wp, 1.0_wp, 1.0_wp, 1.0_wp, 3.0_wp/) 
   RHS1 = (/2.0_wp, 3.0_wp, 4.0_wp, 5.0_wp, 7.0_wp, 8.0_wp, 10.0_wp/) 
   RHS2 = (/5.0_wp, 5.0_wp, 4.0_wp, 5.0_wp, 7.0_wp, 12.0_wp, 12.0_wp, 4.0_wp/) 
   RHS3 = (/3.0_wp, 5.0_wp, 4.0_wp, 5.0_wp, 6.0_wp, 6.0_wp, 2.0_wp/) 
!  First system 
   status = 1 
   DO 
     CALL SCU_factorize( mat, data, VECTOR, status, info ) 
     IF ( status <= 0 ) EXIT 
     DO i = 1, n 
        VECTOR( i ) = VECTOR( i ) / DBLE( FLOAT( i ) ) 
     END DO 
   END DO 
   WRITE( 6, "( /, ' On exit from SCU_factorize,  status = ', I3 )" ) status 
   IF ( status < 0 ) STOP 
   status = 1 
   DO 
     CALL SCU_solve( mat, data, RHS1, X1, VECTOR, status ) 
     IF ( status <= 0 ) EXIT 
     DO i = 1, n        !  multiply by the inverse of A 
        VECTOR( i ) = VECTOR( i ) / DBLE( FLOAT( i ) ) 
     END DO 
   END DO 
   WRITE( 6, "( ' On exit from SCU_solve,      status = ', I3 )" ) status 
   IF ( status < 0 ) STOP 
   WRITE( 6, "( /, ' Solution (first system)', /, ( 8ES9.2 ) )" ) X1( : ) 
!  Second system 
   mat%BD_row( 10 : 12 ) = (/ 1, 6, 8 /) 
   mat%BD_val( 10 : 12 ) = (/ 1.0_wp, 1.0_wp, 1.0_wp /) 
   mat%BD_col_start( 4 ) = 13 
   mat%CD_col( 10 ) = 1 
   mat%CD_val( 10 ) = 1.0_wp 
   mat%CD_row_start( 4 ) = 11 
   status = 1 
   DO 
     CALL SCU_append( mat, data, VECTOR, status, info ) 
     IF ( status <= 0 ) EXIT 
     DO i = 1, n 
        VECTOR( i ) = VECTOR( i ) / DBLE( FLOAT( i ) ) 
     END DO 
   END DO 
   WRITE( 6, "( /, ' On exit from SCU_append,     status = ', I3 )" ) status 
   IF ( status < 0 ) STOP 
   status = 1 
   DO 
     CALL SCU_solve( mat, data, RHS2, X2, VECTOR, status ) 
     IF ( status <= 0 ) EXIT 
     DO i = 1, n 
        VECTOR( i ) = VECTOR( i ) / DBLE( FLOAT( i ) ) 
     END DO 
   END DO 
   WRITE( 6, "( ' On exit from SCU_solve,      status = ', I3 )" ) status 
   IF ( status < 0 ) STOP 
   WRITE( 6, "( /, ' Solution (second system)', /, ( 8ES9.2 ) )" ) X2 
!  Third system 
   row_del = 1 
   col_del = 2 
   status = 1 
   CALL SCU_delete( mat, data, VECTOR, status, info, col_del, & 
                     row_del = row_del ) 
   WRITE( 6, "( /, ' On exit from SCU_delete,     status = ', I3 )" ) status 
   IF ( status < 0 ) STOP 
   status = 1 
   DO 
     CALL SCU_solve( mat, data, RHS3, X3, VECTOR, status ) 
     IF ( status <= 0 ) EXIT 
       DO i = 1, n 
        VECTOR( i ) = VECTOR( i ) / DBLE( FLOAT( i ) ) 
     END DO 
   END DO 
   WRITE( 6, "( ' On exit from SCU_solve,      status = ', I3 )" ) status 
   IF ( status < 0 ) STOP 
   WRITE( 6, "( /, ' Solution (third system)', /, ( 8ES9.2 ) )" ) X3 
   CALL SCU_terminate( data, status, info ) 
   WRITE( 6, "( /, ' On exit from SCU_terminate,  status = ', I3 )" ) status 
   END PROGRAM GALAHAD_SCU_EXAMPLE 
\end{verbatim}
}
This produces the following output:
{\tt
\begin{verbatim}
 On exit from SCU_factorize,  status =   0
 On exit from SCU_solve,      status =   0

 Solution (first system)
 1.00E+00 1.00E+00 1.00E+00 1.00E+00 1.00E+00 1.00E+00 1.00E+00

 On exit from SCU_append,     status =   0
 On exit from SCU_solve,      status =   0

 Solution (second system)
 3.00E+00 2.00E+00 1.00E+00 1.00E+00 1.00E+00 1.00E+00 1.00E+00 1.00E+00

 On exit from SCU_delete,     status =   0
 On exit from SCU_solve,      status =   0

 Solution (third system)
 1.00E+00 2.00E+00 1.00E+00 1.00E+00 1.00E+00 1.00E+00 1.00E+00

 On exit from SCU_terminate,  status =   0
\end{verbatim}
}

\end{document}

